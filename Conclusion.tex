\phantomsection
\addcontentsline{toc}{chapter}{Conclusion générale}
\chapter*{Conclusion générale}

Sur le plan théorique, notre travail s'est situé dans le cadre du traitement des connaissances incertaines,
principalement sur la fusion de croyances sous la théorie de Dempster-Shafer. Nous avons également pris connaissances
de plusieurs théories de l'incertain et de l'imprécision.

Nous avons donc fait une étude pour comprendre et pouvoir exposer les notions générales de la théorie de Dempster-Shafer,
en expliquant les fonctions de croyance, les règles de combinaisons et les règles de décision. 
Nous avons aussi conçu et implémenté les algorithmes nécessaires pour l'exploitation complète de cette théorie. De plus,
nous avons présenté brièvement les autres théories de l'incertain qui sont situées dans le même domaine que notre sujet.

Sur le plan pratique, nous avons réalisé deux applications. La première est un logiciel qui permet la représentation,
puis la résolution d'un problème d'incertitude dans un système multi-agents en appliquant la théorie de Dempster-Shafer.
La deuxième application est une plateforme qui fait appel à différent outils qui implémentent des théories de
l'intelligence artificielle.

Comme perspectives, il serait intéressant de modifier notre plateforme en ajoutant des interfaces pour d'autres outils
manquants dans cette version, ou même de supprimer les existantes s'ils ne seront plus utiles. Pour cela, nous avons créé
la plateforme de façon à faciliter son extention.

Nous souhaitons que notre travail soit bénéfique pour notre université et nous espérons qu'il sera utilisé dans autres
établissements. En effet, il peut fournir une aide importante pour les personnes spécialisées dans le domaine de
l'intelligence artificielle.