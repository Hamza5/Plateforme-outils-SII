\phantomsection
\addcontentsline{toc}{chapter}{Conclusion générale}
\chapter*{Conclusion générale}

Sur le plan théorique, notre travail se situe dans le cadre de traitement des connaissances incertaines,
principalement, sur la fusion de croyances sous la théorie de Dempster-Shafer. En plus, il comporte d'autres
théories d'incertain.

Nous avons fait une étude pour comprendre et pouvoir exposer les notions générales de la théorie de Dempster-Shafer,
en expliquant les fonctions de croyance, les règles de combinaisons et les règles de décision. 
Nous avons aussi conçu et implémenté les algorithmes nécessaires pour l'exploitation complète de cette théorie. De plus,
nous avons présenté brièvement les autres théories d'incertain qui sont situées dans le même domaine de notre sujet.

Sur le plan pratique, nous avons réalisé deux applications. La première est un logiciel qui permet la représentation
et puis la résolution d'un problème d'incertitude dans un système multi-agents en appliquant la théorie de Dempster-Shafer.
La deuxième application est une plateforme qui fait appel à des différent outils qui implémentent des théories de
l'intélligence artificielle.

Comme perspectives, il serait intéressent de modifier notre plateforme en ajoutant des interfaces pour d'autres outils
manquantes dans cette version, ou même de supprimer les existantes s'ils ne seront plus utiles. Pour cela, nous avons créé
la plateforme d'une façon à faciliter sa extention.

Nous souhaitons que notre travail soit bénéfique pour notre université et nous espérons qu'il sera utilisé dans autres
établissements. Il peut fournir un aide important pour les gens spécialisés dans le domaine de l'intélligence artificielle.