\documentclass[a4paper, 12pt]{report}
\usepackage[utf8]{inputenc}
\usepackage[T1]{fontenc}
\usepackage{lmodern}
\usepackage[french]{babel}
\usepackage[top=1 cm, left=2 cm, right=3 cm]{geometry}
\usepackage{graphicx}


\usepackage{fancyhdr}
\begin{document}
\begin{center}
\vspace*{4em}
\section*{Résumé}
\end{center}
\thispagestyle{empty}
\subparagraph{}
Ce projet se situe dans le contexte de la manipulation des connaissances incertaines en intelligence artificiel. Il comporte
deux parties principales.
\subparagraph{}
La conception d'une application permettant de fusioner des connaissances incertaines utilisant les règles de combinaison développées dans le cadre de la théorie des fonctions de croyance.
\subparagraph{}
La réalisation d'une interface intégrant différents outils pour représenter et pour raisonner sur des connaissances
incertaines en utilisant les théories autour de la théorie des possibilités.\\[2em]
  
\textbf{ Mots-clés :} théorie de fonctions de croyances, Dempster et Shafer, fusion d'informations, traitement de l'incertain,
théories de possibilité.
\begin{center}
\vspace{2em}
\section*{Abstract}
\end{center}
\subparagraph{}
This project is situated in the context of the handling of uncertain knowledge in artificial intelligence. It includes
two main parts.
\subparagraph{}
The design of an application allowing the merging of uncertain knowledges using the combination rules developed within the
framework of the belief functions theory.
\subparagraph{}
The realization of an interface integrating various tools for representing and reasoning on uncertain knowledge using the
theories around the possibility theory.\\[2em]

\textbf{Keywords :} belief functions theory, Dempster and Shafer, informations merging, uncertain treatment, possibility
theories.
\end{document}