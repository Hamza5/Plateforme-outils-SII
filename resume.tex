\documentclass[a4paper, 12pt]{report}
\usepackage[utf8]{inputenc}
\usepackage[T1]{fontenc}
\usepackage{lmodern}
\usepackage[french]{babel}
\usepackage[top=1 cm, left=2 cm, right=3 cm]{geometry}
\usepackage{graphicx}


\usepackage{fancyhdr}
\begin{document}
\begin{center}
\vspace*{4em}
\section*{Résumé}
\end{center}
\thispagestyle{empty}
\subparagraph{}
Notre travaille est réalisé dans les cadre de faciliter la manipulation de connaissances incertaines en intelligence artificiel, Il présente deux parties principales.
\subparagraph{}
La conception d'une application permettant l'utilisation de la fusion de connaissances pour le calcul de la décision. Plus précisément, l'application s'intéresse à la fusion utilisant les règles de fonction de croyance, développés dans le cadre de la théorie des fonction de croyance qui permettent de fusionner explicitement, à partir d'outils mathématiques, l'incertitude liée aux connaissances. 
\subparagraph{}
La réalisation d'une interface intégrante différentes outils pour présenter et pour résonner sur des connaissances incertaines en utilisation les théories autour de la théorie des possibilités.  \vspace{2em}
  
\textbf{ Mots-clés} :théorie des fonctions de croyances, Dempster et Shafer, fusion d'informations, traitement de l'incertain, théories de possibilité.
\begin{center}
\vspace{2em}
\section*{Abstract}
\end{center}
\subparagraph{}
Our work is done in the context of expedite the handling of uncertain knowledge in artificial intelligence, it has two main parts.
\subparagraph{}
Designing an application for the use of the fusion of knowledge in the calculation of the decision. More specifically, the application focuses on the fusion using the rules of belief function, developed in the context of the belief function theory that allow to merge explicitly, using mathematical tools, the uncertainty of knowledge.
\subparagraph{}
the relizationof an interface integrating different tools to present and to resonate on uncertain knowledge in use theories about the possibility theory. \vspace{2em}

\textbf{Keywords :} belief function theory, Dempster and Shafer, information fusion, uncertain treatment,  possibility theories.
\end{document}