\documentclass[a4paper, 12pt]{report}
\usepackage[utf8]{inputenc}
\usepackage[T1]{fontenc}
\usepackage{lmodern}
\usepackage[french]{babel}
\usepackage[top=1 cm, left=2 cm, right=3 cm]{geometry}
\usepackage{graphicx}
\usepackage{algorithm}
\usepackage{algorithmic}
\usepackage{fancyhdr}
\begin{document}
\begin{center}
\section*{Résumé}
\end{center}
\thispagestyle{empty}
\subparagraph{}
L'intelligence artificielle consiste à représenter les connaissances, à modéliser et raisonner sur celle-ci. Le problème de décision est inhérent à cette discipline. Il consiste à aider un agent à prendre une décision à partir des connaissances qui sont parfois entachés d'incertitude et issues de diverses sources. 
\subparagraph{}
Ce document décrit l'utilisation de la fusion de connaissances numériques pour le calcul de la décision. Plus précisément, nous nous intéressons à la fusion utilisant la théorie de Dempster et Shafer qui permet de représenter explicitement, à partir d'outils mathématiques, l'incertitude liée aux connaissances. Par cette théorie nous pouvons modéliser le degré de croyance, qui se révèle particulièrement efficace lors de la combinaison multi-sources.
\subparagraph{}
La fusion de connaissance autour de cette théorie passe par plusieurs étapes. Nous nous somme basé dans notre travail à exploiter chacune d'elles et nous avons choisi les méthodes adéquates aux connaissances numériques multi-capteur puis développé des algorithmes permettant la réalisation du schéma de la fusion par la suite  nous avons créé une boite à outils englobant l'implémentation  de ces différents algorithmes. 
\subparagraph{}
Enfin, nous avons étudié les différents domaines utilisant la théorie de Dempster et Shafer et nous avons choisi le domaine de la surveillance de la qualité de l'air pour appliquer nos études  et développé un outil de contrôle et d'estimation de la qualité de l'air autour de cette théorie.
\\
\textbf{   Mots-clés} : incertitude, théorie des fonctions de croyances, Dempster et Shafer, fusion d'informations.
\begin{center}
\section*{Abstract}
\end{center}
\subparagraph{}
The artificial intelligence is to represent knowledge, to model and reason about it. The decision problem is inherent in this discipline. It is to help an agent to make a decision based on knowledge that is sometimes plagued by uncertainty and from various sources.
\subparagraph{}
This document describes the use of the fusion of digital knowledge for the calculation of the decision. Specifically, we focus on the fusion using the Dempster Shafer and used to represent explicitly, using mathematical tools, the uncertainty of knowledge. By this theory we can model the degree of belief, which is particularly effective when combining multiple sources.
\subparagraph{}
The merger of knowledge around this theory goes through several stages. We sum based in our work to exploit each of them and chose the appropriate methods for multi-sensor digital knowledge and developed algorithms for carrying out the scheme by following the merger we have created a toolkit encompassing implementation of these algorithms.
\subparagraph{}
Finally, we studied the different domains using the Dempster and Shafer and we chose the field of monitoring air quality to apply our studies and developed a tool for control and estimation of the quality of air around this theory.
\\
\textbf{Keywords :} uncertainty, belief function theory, Dempster and Shafer, information fusion.
\end{document}