\parskip=0.6em
\chapter{Les théories de l'incertain}

\phantomsection
\addcontentsline{toc}{section}{Introduction}
\section*{Introduction}



\section{L’incertain}

Les approches logiques de l’incertain permettent d’utiliser un langage formel pour la description des connaissances et le raisonnement automatique. Elles constituent une référence aux autres formalismes surtout pour le raisonnement, néanmoins les connaissances ne  sont pas structurées.

les principales représentations des connaissances incertaines sont le mode logique et le mode graphique.
Sur le plan de la représentation le mode graphique est le plus explicite en soumettant les relations de dépendances qui existe entre les différentes variables .sur le plan de résonnement, le mode logique offre une machinerie d’inférence efficace.

Plusieurs chercheurs ont permis l’émergence d’un certain nombre de modèles graphiques offrant un cadre de représentation plus structuré.

La théorie des possibilités est une théorie de l’incertain ayant pour vocation de manipuler des connaissances incomplètes. Elle diffère de la théorie de probabilité  vu quelle manipule deux mesures duales : possibilité et nécessité. Cette théorie a été développée dans deux directions : qualitative et quantitative ceci permet en fait de définir deux types de réseaux causaux possibilistes : les réseaux causaux possibilistes basé sur le minimum (qualitatif) et une autre base sur le produit (quantitative).



\subsection{Théorie de probabilité(BNT)}

La théorie des réseaux bayésien peut être considérée comme une fusion de diagrammes d'incidence et le théorème de Bayes. La probabilité qu'un événement se produise étant donné que un autre événement a déjà eu lieu est appelé une probabilité conditionnelle. 

Un réseau bayésien une représentation probabiliste des relations incertaines, qui se est avéré être utile pour la modélisation de problèmes du monde réel .Le modèle probabiliste est décrit qualitativement par un graphe acyclique orienté, ou DAG (Directed Acyclic Graph). Les sommets du graph représentent des variables, les arcs représentent la dépendance entre les variables. Le réseau comprennent aussi un ensemble de tables de probabilités, en indiquant les probabilités pour les vrais / fausses valeurs des variables.

Les avantages du  modèle graphique est qu’un réseau bayésien peut être utilisé pour apprendre les relations causales, et peut donc être utilisé pour obtenir la compréhension d'un domaine de problème et de prévoir les conséquences de l'intervention.

Ce que le champ a manqué est un logiciel à usage général correspondant. C’est pour cela la tentative de construire un tel paquet, appelé le Bayes Net Boîte à outils (BNT).

Une des plus grandes forces de BNT est qu'il offre une variété d'algorithmes d'inférence, chacun fait différents métiers entre la précision, de la généralité, la simplicité, la vitesse, etc.


\subsection{Théorie de possibilité quantitative}

\subsubsection{Graphique}

Un graphe possibiliste basé sur le produit (quantitatif), noté par GP P, est un graphe possibiliste ou les possibilités conditionnelles sont obtenues par le conditionnement de type produit. La distribution de possibilité des réseaux possibilistes bases sur le produit, notée par Pi P, est obtenue par la règle de chaınage 
pi P (V1, .., VN) = PRODi=1..N pi (Vi/PARVi).
 	
Ou PROD est l’opérateur produit.
La Boîte à outils PNT possède de différents algorithmes pour les réseaux causaux possibilistes basée sur le produit à connexions multiples et pour les polyarbres. 
\subsubsection{Logique}

La logique possibiliste offre un cadre générale pour représenter les connaissances incertaines, en terme de formules logiques classique auxquelles sont associé des pondérations appartenant à une échèle linaire [0,1].
\subsection{Théorie de possibilité qualitative}
\subsubsection{Graphique}

Un graphe possibiliste basé sur le minimum, noté par GPM, est un graphe possibiliste ou les possibilités conditionnelles sont obtenues par le conditionnement minimum. La distribution de possibilité des réseaux possibilistes basée sur le minimum, notée par pi M, est obtenue par la règle de chainage :
 pi M (A1, .., AN) = MINi=1..N pi (Ai/teta Ai) 
Ou MIN est  l’opérateur minimum.

La Boite à outils PNT possède aussi de différents algorithmes pour les réseaux causaux possibilistes basée sur le minimum à connexions multiples et pour les polyarbres. 
\subsubsection{Logique}



\subsection{Décision dans l’incertain}
\subsubsection{Graphique}

\subsubsection{Logique}
La logique possibiliste qualitative est une logique de l'incertain conçue pour raisonner avec des connaissances incomplètes et partiellement inconsistantes [18].

NB : La théorie de Dempster and Shafer  est une théorie de l’incertain nous l’avant louper parce qu’elle expliquée théoriquement en détail dans le chapitre I, et dans le chapitre suivant on va parler sur les algorithmes implémentés sur cette théorie.
\section{L’imprécision}

La logique floue permet de solutionner tous les problèmes où on dispose de connaissances imprécises, soumises à des incertitudes de nature non probabiliste.

cette dernière est une forme de logique polyvalente qui traite l’approximation, plutôt que le raisonnement fixe et exacte. Par rapport à la logique binaire traditionnelle (où les variables peuvent prendre des valeurs vraies ou fausses), les variables de logique floue peuvent avoir une valeur de vérité qui varie en degré entre 0 et 1. La logique floue a été étendue pour gérer le concept de vérité partielle, où la valeur de vérité peut varier entre complètement vrai et complètement faux. 
\subsection{Fuzzy logic}

a Boîte à outils  Fuzzy est une collection de fonctions intégrées sur le MATLAB elle fournit des outils pour permettre de créer et d'éditer systèmes d'inférence floue. Et un bloc Simulink pour l'analyse, la conception et la simulation des systèmes basés sur la logique floue.
La boîte à outils vous permet de modéliser les comportements de systèmes complexes en utilisant des règles logiques simples, puis de mettre en œuvre ces règles dans un système d'inférence floue.

\section{Outils}

Avant de définir l’UBCSAT, nous allons d’abord définir la notion de satisfiabilité en logique (SAT). 
\subsection{SAT}
Soit F une formule propositionnelle sous la forme normale conjonctive (CNF). Le problème SAT est un problème de décision NP-complet qui consiste à déterminer si F admet ou non un modèle [46].

Le problème de satisfiabilité propositionnelle (SAT) est un sujet d'étude important dans de nombreux domaines de l'informatique, SAT est définie par les ces composantes :   
	Soit X={x1, x2,…, xn} un ensemble de n variable booléennes.
    Soit C= {c1, c2,…, cm} un ensemble de m clauses ou :
 	chaque clause est une disjonction de littéraux,
 	chaque littéral est une variable ou sa négation.
    Soit D, la donnée SAT	, composé une conjonction de littéraux.
Le problème SAT consiste à déterminer s’il existe une assignation des variables xi de X telle que la donnée D soit satisfaisante. S’il existe une assignation de variables qui satisfait toutes les clauses le SAT admet une réponse ‘Oui’ ou ‘Non’ sinon. 

UBCSAT (University of British Columbia SAT)

Pour faire preuve de satisfiabilité, il est nécessaire d’utiliser un logiciel.

L'un des défis de l'élaboration du projet UBCSAT était de construire un environnement flexible, riche en fonctionnalités sans compromettre l'efficacité algorithmique. ce programme est performant sur les instances SAT issues de problèmes réels. Il offre la gestion des contraintes pseudo booléennes, et décline un grand nombre de problèmes de décision ou d’optimisation en terme de problème SAT ou pseudo booléen.    
\subsection{WMAXSAT}

toutefois, dans le cas de réponse négative du SAT, pour trouver toutes le nombre maximale de clauses pouvant être satisfaites à la fois. Pour cela le problème du ‘Maximum Satisfiabilité’ ou MAXSAT a été définie. Max-SAT est la version optimale de la SAT dont le but est de satisfaire le nombre maximal de clauses.

Le problème avec le MAXSAT c’est qu’elle associe à toutes les clauses le même poids, d’où la définition du problème du MAXSAT pondéré ou WMAXSAT qui permet à attribuer des poids aux différentes clauses pour spécifier les clauses simultanément satisfaites par augmenter la somme de leurs poids et l’affaiblir pour les clauses insatisfaites.

Actuellement, UBCSAT comprend des implémentations de deux algorithmes conçu pour supporter les versions MAX-SAT, ainsi que pondéré MAX-SAT.

%\phantomsection
%\addcontentsline{toc}{section}{Conclusion}
\section*{Conclusion}
