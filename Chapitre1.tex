\chapter{Théorie de Dempster and Shafer}
\phantomsection
\addcontentsline{toc}{section}{Introduction}

\section*{Introduction}

La théorie de Dempster-Shafer, également connu comme la théorie de l'évidence,
ou la théorie des fonctions de croyance, est une théorie mathématique du raisonnement
sur l’évidence et la plausibilité. Elle a été développée par Glenn Shafer (1976)
en basant sur des travaux antérieurs de Arthur Dempster (1968). La théorie a attirer
l’attention des chercheurs de l'intelligence artificielle au début des années 80.

Dans un espace discret finis, la théorie de Dempster-Shafer représente la généralisation
de la théorie des probabilités où les les probabilités sont assignés à des ensembles
contrairement aux singletons mutuellement exclusifs.

\section{Concepts de base}

\subsection{Quelques notations}

Dans cette section, Nous allons présenter les conventions de notation utilisées dans le reste de ce mémoire.

Soit $\Omega$ un univers (Un ensemble fini qui contient tout les propositions auxquels on s'intéresse),
il est appelé également le cadre de discernement. On note $\varnothing$ l’ensemble qui
ne contient aucun élément de $\Omega$. $\mathcal{P}(\Omega)$ représente l’ensemble contenant tous
les parties (les sous-ensembles) de $\Omega$. On note aussi les opérations binaires
sur les ensembles $\subset$, $\cup$ et $\cap$, l’inclusion, l’union et l’intersection,
respectivement. On appel hypothèse un élément de $\mathcal{P}(\Omega)$.

La théorie de de Dempster-Shafer est caractérisée par trois fonctions principales : \textbf{l’assignement
de probabilité de base}, \textbf{la croyance} et \textbf{la plausibilité}.

\subsection{L’assignement de probabilité de base}

L’assignement de probabilité de base (\emph{Basic probability assignement bpa}),
appelé aussi la fonction de masse, noté $m$, est une fonction qui affecte le degré d’évidence disponible à
un élément $A$, et seulement à $A$, de $\mathcal{P}(\Omega)$ dans l’intervalle $[0,1]$, tel que
les $m(A)$ s’additionnent à $1$. $m(\varnothing)$ --qui représente l’absence de solution-- est
toujours égale à $0$ car $\Omega$ doit être exhaustive, et la somme de $m(A)$ vaut $1$. Chaque
élément A tel que $m(A) > 0$ est appelé un élément focal.

Généralement, le bpa n’est pas un équivalent de la fonction de probabilités classique. En effet,
la valeur exacte de la probabilité (dans le sens classique) appartient à un intervalle borné par
deux valeurs qui sont \emph{la croyance} et \emph{la plausibilité}.\\
Formellement, on peut représenter ça par :
\begin{equation}
m : \mathcal{P}(\Omega) \mapsto [0,1]
\end{equation}
\begin{equation}
m(\varnothing) = 0
\end{equation}
\begin{equation} \label{somme_masses}
\sum_{A \in \mathcal{P}(\Omega)} m(A) = 1
\end{equation}

\subsubsection{Exemple}
Dans une enquête d’un incident de vol, trois personnes \textit{P1}, \textit{P2}, \textit{P3} sont
accusés. L’enquêteur pense à 30\% que P1 est le voleur, à 50\% que l’un de P2 ou P3 soit le voleur,
et à 10\% que l’un de P1 ou P3 le soit.\\
Le cadre de discernement $\Omega = \{P1, P2, P3\}$.\\
$\mathcal{P}(\Omega) = \{\varnothing, \{P1\}, \{P2\}, \{P3\}, \{P1, P2\}, \{P1, P3\}, \{P2, P3\},
\Omega\}$.\\La distribution des masses : $m(\{P1\}) = 0.3$, $m(\{P2, P3\})  = 0.5$, $m(\{P1, P3\}) = 0.1$
et $m(\Omega) = 1 - (0.3+0.5+0.1) = 0.1$ à partir de l’équation \ref{somme_masses}. Les masses des
hypothèses restantes sont tout égaux à $0$.

\subsection{La croyance}

La croyance (Belief) représente la quantité du confiance qui supporte une hypothèse $A$ de 
$\mathcal{P}(\Omega)$ y compris tout ses parties. On la note $Cr(A$) ou plus souvent $Bel(A)$. Il est
claire que $Bel(\varnothing)$ est nul, et $Bel(\Omega)$ est toujours 1. Noter que la croyance est une
mesure non additive, c’est à dire que $Bel(A) + Bel(\Omega - A) \neq 1$.\\
Formellement :
\begin{equation}
Bel(\varnothing)=0
\end{equation}
\begin{equation}
Bel(\Omega)=1
\end{equation}
\begin{equation}
Bel(A) = \sum_{B \slash B \subseteq A} m(B)
\end{equation}
\begin{equation}
\forall p,q \in \mathcal{P}(\Omega) \medskip Bel(p \cup q) \geq Bel(p) + Bel(q) - Bel(p \cap q)
\end{equation}
Comme on peut obtenir le bpa avec la fonction inverse suivante :
\begin{equation}
m(A) = \sum_{B \slash B \subseteq A} (-1)^{|A-B|} Bel(B)
\end{equation}
Où $|A-B|$ est la différence des cardinalités entre les deux ensembles.

\subsubsection{Exemple}
Continuant de l'exemple précédent, on calcule la croyance de chaque hypothèse :

\begin{table}[h]
\begin{center}
\begin{tabular}{|l|l|}
\hline
Hypothèse $A$ & Croyance $Bel(A)$\\
\hline
$\varnothing$ & $0$ \\
\hline
$\{P1\}$ & $0.3$ \\
\hline
$\{P2\}$ & $0$ \\
\hline
$\{P3\}$ & $0$ \\
\hline
$\{P1, P2\}$ & $0.3 + 0 + 0 = 0.3$ \\
\hline
$\{P1, P3\}$ & $0.3 + 0 + 0.1 = 0.4$ \\
\hline
$\{P2, P3\}$ & $0 + 0 + 0.1 = 0.1$ \\
\hline
$\Omega$ & $1$ \\
\hline
\end{tabular}
\caption{Les croyances calculées à partir de la distribution de masses}
\end{center}
\end{table}