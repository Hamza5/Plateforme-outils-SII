\chapter{Théorie de Dempster and Shafer}
\phantomsection
\addcontentsline{toc}{section}{Introduction}

\section*{Introduction}

La théorie de Dempster-Shafer, également connu comme la théorie de l'évidence,
ou la théorie des fonctions de croyance, est une théorie mathématique du raisonnement
sur l’évidence et la plausibilité. Elle a été développée par Glenn Shafer (1976)
en basant sur des travaux antérieurs de Arthur Dempster (1968). La théorie a attirer
l’attention des chercheurs de l'intelligence artificielle au début des années 80.

Dans un espace discret finis, la théorie de Dempster-Shafer représente la généralisation
de la théorie des probabilités où les les probabilités sont assignés à des ensembles
contrairement aux singletons mutuellement exclusifs.

\section{Concepts de base}

\subsection{Quelques notations}

Dans cette section, Nous allons présenter les conventions de notation utilisées dans le reste de ce mémoire.

Soit $\Omega$ un univers (Un ensemble fini qui contient tout les propositions auxquels on s'intéresse),
il est appelé également le cadre de discernement. On note $\varnothing$ l’ensemble qui
ne contient aucun élément de $\Omega$. $\mathcal{P}(\Omega)$ représente l’ensemble contenant tous
les parties (les sous-ensembles) de $\Omega$. On note aussi les opérations binaires
sur les ensembles $\subset$, $\cup$ et $\cap$, l’inclusion, l’union et l’intersection,
respectivement. On appel hypothèse un élément de $\mathcal{P}(\Omega)$.

La théorie de de Dempster-Shafer est caractérisée par trois fonctions principales : \textbf{l’assignement
de probabilité de base}, \textbf{la croyance} et \textbf{la plausibilité}.

\subsection{L’assignement de probabilité de base}

L’assignement de probabilité de base (\emph{Basic probability assignement bpa}),
appelé aussi la fonction de masse, noté $m$, est une fonction qui affecte le degré d’évidence disponible à
un élément $A$, et seulement à $A$, de $\mathcal{P}(\Omega)$ dans l’intervalle $[0,1]$, tel que
les $m(A)$ s’additionnent à $1$. $m(\varnothing)$ --qui représente l’absence de solution-- est
toujours égale à $0$ car $\Omega$ doit être exhaustive, et la somme de $m(A)$ vaut $1$. Chaque
élément A tel que $m(A) > 0$ est appelé un élément focal.

Généralement, le bpa n’est pas un équivalent de la fonction de probabilités classique. En effet,
la valeur exacte de la probabilité (dans le sens classique) appartient à un intervalle borné par
deux valeurs qui sont \emph{la croyance} et \emph{la plausibilité}.\\
Formellement, on peut représenter ça par :
\begin{equation}
m : \mathcal{P}(\Omega) \mapsto [0,1]
\end{equation}
\begin{equation}
m(\varnothing) = 0
\end{equation}
\begin{equation} \label{somme_masses}
\sum_{A \in \mathcal{P}(\Omega)} m(A) = 1
\end{equation}

\subsubsection{Exemple}
Dans une enquête d’un incident de vol, trois personnes \textit{P1}, \textit{P2}, \textit{P3}
sont accusés. L’enquêteur pense à 30\% que P1 est le voleur, à 50\% que l’un de P2 ou P3 soit
le voleur, et à 10\% que l’un de P1 ou P3 le soit.\\
Le cadre de discernement $\Omega = \{P1, P2, P3\}$.\\
$\mathcal{P}(\Omega) = \{\varnothing, \{P1\}, \{P2\}, \{P3\}, \{P1, P2\}, \{P1, P3\},
\{P2, P3\}, \Omega\}$.\\
La distribution des masses : $m(\{P1\}) = 0.3$, $m(\{P2, P3\})  = 0.5$,
$m(\{P1, P3\}) = 0.1$ et $m(\Omega) = 1 - (0.3+0.5+0.1) = 0.1$ à partir de l’équation
\ref{somme_masses}. Les masses des hypothèses restantes sont tout égaux à $0$.

\subsection{La croyance}

La croyance (Belief) représente la quantité du confiance qui supporte une hypothèse $A$ de 
$\mathcal{P}(\Omega)$ y compris tout ses parties. On la note $Cr(A$) ou plus souvent $Bel(A)$. Il est
claire que $Bel(\varnothing)$ est nul, et $Bel(\Omega)$ est toujours 1. Noter que la croyance est une
mesure non additive, c’est à dire que $Bel(A) + Bel(\Omega - A) \neq 1$.\\
Formellement :
\begin{equation}
Bel(\varnothing)=0
\end{equation}
\begin{equation}
Bel(\Omega)=1
\end{equation}
\begin{equation}
Bel(A) = \sum_{B \slash B \subseteq A} m(B)
\end{equation}
\begin{equation}
\forall p,q \in \mathcal{P}(\Omega) \medskip Bel(p \cup q) \geq Bel(p) + Bel(q) - Bel(p \cap q)
\end{equation}
Comme on peut obtenir le bpa avec la fonction inverse suivante :
\begin{equation}
m(A) = \sum_{B \slash B \subseteq A} (-1)^{|A-B|} Bel(B)
\end{equation}
Où $|A-B|$ est la différence des cardinalités entre les deux ensembles.

\subsubsection{Exemple}
Continuant de l'exemple précédent, on calcule la croyance de chaque hypothèse :

\begin{table}[h]
\begin{center}
\begin{tabular}{|l|l|}
\hline
Hypothèse $A$ & Croyance $Bel(A)$\\
\hline
$\varnothing$ & $0$ \\
\hline
$\{P1\}$ & $0.3$ \\
\hline
$\{P2\}$ & $0$ \\
\hline
$\{P3\}$ & $0$ \\
\hline
$\{P1, P2\}$ & $0.3 + 0 + 0 = 0.3$ \\
\hline
$\{P1, P3\}$ & $0.3 + 0 + 0.1 = 0.4$ \\
\hline
$\{P2, P3\}$ & $0 + 0 + 0.1 = 0.1$ \\
\hline
$\Omega$ & $1$ \\
\hline
\end{tabular}
\caption{Les croyances calculées à partir de la distribution de masses}
\end{center}
\end{table}

\subsection{La plausibilité}

La plausibilité (Plausibility) représente la croyance qu’elle soit possible
d’être affectée à une hypothèse $A$ de $\mathcal{P}(\Omega)$, autrement dit
: c’est la croyance affectée à A et aux hypothèses dont leurs intersection
avec $A$ n’est pas vide. On la note $Pl(A)$. La valeur de la plausibilité est
toujours supérieur ou égale à celle de la croyance. Identiquement à la
croyance, $Pl(\varnothing)$ est nul, et $Pl(\Omega)$ vaut $1$.\\
Formellement :
\begin{equation}
Pl(\varnothing) = 0
\end{equation}
\begin{equation}
Pl(\Omega) = 1
\end{equation}
\begin{equation}
Pl(A) = \sum_{B \slash B \cap A \neq \varnothing} m(B)
\end{equation}
La plausibilité peut aussi être dérivée à partir de la croyance:
\begin{equation}
Pl(A) = 1 - Bel(\bar{A})
\end{equation}
Où $A$ est le complément de $A$ dans $\Omega$

La probabilité classique d’un évènement $A$, $P(A)$ est représentée par
l’intervalle qui a comme borne inférieur et supérieur $Bel(A)$ et $Pl(A)$,
respectivement, c’est à dire $Pl(A) \leq P(A) \leq Bel(A)$. Si $Bel(A) = Pl(A)$
alors la probabilité de $A$ est unique et $P(A) = Bel(A) = Pl(A)$.

\subsubsection{Exemple}
On calcule les plausibilités pour le même exemple:

\begin{table}[h]
\begin{center}
\begin{tabular}{|l|l|}
\hline
Hypothèse $A$ & Plausibilité $Pl(A)$\\
\hline
$\varnothing$ & $0$ \\
\hline
$\{P1\}$ & $0.3 + 0.1 + 0.1 = 0.5$ \\
\hline
$\{P2\}$ & $0.5 + 0.1 = 0.6$ \\
\hline
$\{P3\}$ & $0.5 + 0.1 + 0.1 = 0.6$ \\
\hline
$\{P1, P2\}$ & $0.3 + 0.5 + 0.1 + 0.1 = 1$ \\
\hline
$\{P1, P3\}$ & $0.3 + 0.5 + 0.1 + 0.1 = 1$ \\
\hline
$\{P2, P3\}$ & $0.5 + 0.1 + 0.1 = 0.7$ \\
\hline
$\Omega$ & $1$ \\
\hline
\end{tabular}
\caption{Les plausibilités calculées à partir de la distribution de masses}
\end{center}
\end{table}
\section{Règles de combinaison}

Souvent, on obtient l’information à partir des sources distincts et indépendantes,
et ces sources ne sont pas parfaites généralement, c’est pour ça on a besoin de
fusionner les évidences fournies par ces sources en utilisant des règles de
combinaison. Il existe trois types de combinaison : \emph{conjonctive},
\emph{disjonctive}, et \emph{mixte}.

\subsection{Combinaison conjonctive}

Dempster à introduit une règle connue sous le nom de la règle de combinaison
de Dempster (\emph{Dempster's Combination Rule}) pour fusionner deux pièces
d’évidence qui peuvent être représentées par deux fonctions de croyance.Cette
règle respecte les quatre propriétés algébriques suivantes : l’associativité,
la commutativité, l’idempotence et la continuité.

Soient deux fonctions de masse $m_1(A)$ et $m_2(A)$ associés à deux fonctions
de croyances $Bel_1(A)$ et $Bel_2(A)$ respectivement, $(Bel_1 \oplus Bel_2)(A)$
est définie à travers la masse $m_{1 \oplus 2}(A)$ comme suit :
\begin{equation}
m_{1 \oplus 2}(A) = \sum_{B \cap C = A} m_1(B) m_2(C)
\end{equation}

Néanmoins, cette règle a été critiqué par de nombreux chercheurs, parce qu’elle
n’est pas normalisée, en d’autre termes, elle permet d’affecter des masses à
des ensembles vides, ce problème est dû au fait que l’intersection de deux
éléments focaux peut générer $\varnothing$, cela dénote le conflit entre les croyances.

Pour corriger ce problème, Shafer a proposé d’ignorer carrément les conflits
et étendre la distribution des masses. Le facteur de conflit est une valeur
$K < 1$ et définie par :
\begin{equation}
K = \sum_{B \cap C = \varnothing} m_1(B) m_2(C)
\end{equation}
Afin de normaliser la distribution de masse la règle de Dempster doit être
multipliée par la quantité $(1 - K)^{-1}$. Donc la règle de Dempster-Shafer
est représentée par l’équation suivante :
\begin{equation}
m(\varnothing) = 0
\end{equation}
\begin{equation}
m(A) = \frac{1}{1-K} \sum_{B \cap C = A \neq \varnothing} m_1(B) m_2(C)
\end{equation}
Cependant, la règle de Shafer n’est pas toujours satisfaisante à cause de la
normalisation qui considère que le conflit vient de l’imperfection des sources.

D’autres chercheurs ont proposé des règles différents, comme Smets qui a introduit
une forme non normalisée pour prendre en compte les univers ouverts. Smets suppose
que le conflit vient du fait que l’ensemble $\Omega$ n’est pas exhaustif. Pour
cette raison, la règle de Smets affecte une masse $K$ à $\varnothing$ qui représente
une hypothèse manquante. Elle est définie par ces équations :
\begin{equation}
K = m(\varnothing) = \sum_{B \cap C = \varnothing} m_1(B) m_2(C)
\end{equation}
\begin{equation}
m(A) = \sum_{B \cap C = A \neq \varnothing} m_1(B) m_2(C)
\end{equation}

Par contre, Yager propose un modèle d’un univers fermé ($\Omega$ est exhaustif).
La mesure de conflit est ajouté à la mesure du cadre de discernement $\Omega$.
Donc le conflit sera transformé en ignorance. On obtient ces équations :
\begin{equation}
m(\varnothing) = \sum_{B \cap C = \varnothing} m_1(B) m_2(C)
\end{equation}
\begin{equation}
m(\Omega) = \left(1 - \sum_{A \in \mathcal{P}(\Omega)}
\sum_{B \cap C = A \neq \varnothing} m_1(B) m_2(C)\right) +
\sum_{B \cap C = \varnothing} m_1(B) m_2(C)
\end{equation}

\subsection{Combinaison disjonctive}

Dans la combinaison disjonctive on considère l’utilisation de l’union à la place
de l’intersection. Les éléments focaux sont obtenus à partir de la table d’union.
La fonction de masse de la règle de combinaison disjonctive est formalisé par
cette équation :
\begin{equation}
m(A) = \sum_{B \cup C = A \neq \varnothing} m_1(B) m_2(C)
\end{equation}
Comme l’union de deux éléments focaux ne peut être jamais vide alors on est sûr
que $m(\varnothing)=0$. Cela veut dire qu’il est impossible d’obtenir un conflit
lors de combinaison. En contrepartie, elle peut causer une perte de spécificité
vu que les masses sont élargis. Cette approche apparaît plus intéressante à
l’absence des informations nécessaires sur la fiabilité des sources.

\subsection{Combinaison mixte}

Dubois et Prade ont proposé une combinaison mixte comme un compromis entre la
combinaison conjonctive et la combinaison disjonctive comme un essai de conserver
les avantages de ces deux. De même que la combinaison de  Yager, le modèle de Dubois
et Prade suppose que le conflit est dû à la non fiabilité des sources. La règle de
Dubois et Prade est définie comme suit :
\begin{equation}
m(A) = \sum_{B \cap C = A \neq \varnothing} m_1(B) m_2(C) +
\sum_{\substack{B \cup C = A \neq \varnothing \\ B \cap C = \varnothing}} m_1(B) m_2(C)
\end{equation}