\parskip=0.6em
\chapter{Théorie de Dempster and Shafer}

\phantomsection
\addcontentsline{toc}{section}{Introduction}
\section*{Introduction}

La théorie de Dempster-Shafer, également connue comme la théorie de l'évidence,
ou théorie des fonctions de croyance, est une théorie mathématique du raisonnement
sur l’évidence et la plausibilité. Elle a été développée par Glenn Shafer (1976)
en se basant sur des travaux antérieurs de Arthur Dempster (1968). Elle a attiré
l’attention des chercheurs de l'intelligence artificielle au début des années 80.

Dans un espace discret fini, la théorie de Dempster-Shafer représente la généralisation
de la théorie des probabilités dans laquelle les probabilités sont assignées à des ensembles,
contrairement aux singletons mutuellement exclusifs.

\section{Concepts de base}

\subsection{Quelques notations}

Nous allons présenter ici les conventions de notation utilisées dans la suite de ce mémoire.

Soit $\Omega$ un univers constitué d'un ensemble fini qui contient toutes les propositions
auxquelles on s'intéresse, appelé également cadre de discernement. On note $\varnothing$
l’ensemble qui ne contient aucun élément de $\Omega$. $\mathcal{P}(\Omega)$ représente
l’ensemble contenant toutes les parties (les sous-ensembles) de $\Omega$. On note aussi
les opérations binaires sur les ensembles $\subset$, $\cup$ et $\cap$, qui sont l’inclusion,
l’union et l’intersection, respectivement. On appelle hypothèse un élément de $\mathcal{P}(\Omega)$.

La théorie de Dempster-Shafer est caractérisée par trois fonctions principales :
\textbf{l’assignement de probabilité de base} (\emph{basic probability assignement bpa}),
\textbf{la croyance} (\emph{belief}) et \textbf{la plausibilité} (\emph{plausibility}).

\subsection{L’assignement de probabilité de base}

Appelé aussi la fonction de masse, noté $m$, c'est une fonction qui affecte le degré d’évidence
disponible à un élément $A$, et seulement à $A$, de $\mathcal{P}(\Omega)$ dans l’intervalle
$[0,1]$, tel que les $m(A)$ s’additionnent à $1$.

$m(\varnothing)$ --qui représente l’absence
de solution-- est toujours égale à $0$ car $\Omega$ doit être exhaustive, et la somme de $m(A)$
vaut $1$. Chaque élément A tel que $m(A) > 0$ est appelé élément focal.

Généralement, le bpa n’est pas un équivalent de la fonction de probabilités classique. En effet,
la valeur exacte de la probabilité (dans le sens classique) appartient à un intervalle borné par
deux valeurs qui sont \emph{la croyance} et \emph{la plausibilité}.

Formellement, on peut représenter cela par :
\begin{equation}
m : \mathcal{P}(\Omega) \mapsto [0,1]
\end{equation}
\begin{equation}
m(\varnothing) = 0
\end{equation}
\begin{equation} \label{somme_masses}
\sum_{A \in \mathcal{P}(\Omega)} m(A) = 1
\end{equation}

\subsubsection{Exemple}
Dans une enquête sur un incident de vol, trois personnes \textit{P1}, \textit{P2}, \textit{P3}
sont accusées. L’enquêteur pense à 30\% que P1 est le voleur, à 50\% que l’un des deux autres
personnes P2 ou P3, est le voleur, et à 10\% que P1 ou P3 le soit.\\
Le cadre de discernement s'écrit ainsi : $$\Omega = \{P1, P2, P3\}.$$
L'ensemble de parties de $\Omega$ est : $$\mathcal{P}(\Omega) = \{\varnothing, \{P1\}, \{P2\},
\{P3\}, \{P1, P2\}, \{P1, P3\}, \{P2, P3\}, \Omega\}$$
La distribution des masses s'exprime ainsi : $$m(\{P1\}) = 0.3, m(\{P2, P3\})  = 0.5,
m(\{P1, P3\}) = 0.1$$ et $$m(\Omega) = 1 - (0.3+0.5+0.1) = 0.1$$ à partir de l’équation \ref{somme_masses}.\\
Les masses des hypothèses restantes sont toutes égalles à $0$.

\subsection{La croyance}

La croyance est la quantité de confiance qui supporte une hypothèse $A$ de
$\mathcal{P}(\Omega)$, toutes parties comprises.\\
On la note $Cr(A$) ou plus souvent $Bel(A)$. Il est clair que $Bel(\varnothing)$
est nulle, et $Bel(\Omega)$ est toujours 1. Il faut noter que la croyance est une
mesure non additive, c’est-à-dire que $Bel(A) + Bel(\Omega - A) \neq 1$.

Formellement :
\begin{equation}
Bel(\varnothing)=0
\end{equation}
\begin{equation}
Bel(\Omega)=1
\end{equation}
\begin{equation}
Bel(A) = \sum_{B \slash B \subseteq A} m(B)
\end{equation}
\begin{equation}
\forall p,q \in \mathcal{P}(\Omega) \medskip Bel(p \cup q) \geq Bel(p) + Bel(q) - Bel(p \cap q).
\end{equation}
On peut aussi obtenir le bpa avec la fonction inverse suivante :
\begin{equation}
m(A) = \sum_{B \slash B \subseteq A} (-1)^{|A-B|} Bel(B),
\end{equation}
où $|A-B|$ est la différence des cardinalités entre les deux ensembles.

\subsubsection{Exemple}
En continuité de l'exemple précédent, on calcule la croyance de chaque hypothèse :

\begin{table}[h!]
\centering
\begin{tabular}{|l|l|}
\hline
Hypothèse $A$ & Croyance $Bel(A)$\\
\hline
$\varnothing$ & $0$ \\
\hline
$\{P1\}$ & $0.3$ \\
\hline
$\{P2\}$ & $0$ \\
\hline
$\{P3\}$ & $0$ \\
\hline
$\{P1, P2\}$ & $0.3 + 0 + 0 = 0.3$ \\
\hline
$\{P1, P3\}$ & $0.3 + 0 + 0.1 = 0.4$ \\
\hline
$\{P2, P3\}$ & $0 + 0 + 0.1 = 0.1$ \\
\hline
$\Omega$ & $1$ \\
\hline
\end{tabular}
\caption{Les croyances calculées à partir de la distribution de masses}
\end{table}

\subsection{La plausibilité}

La plausibilité représente la croyance qu’il est possible
d’affecté à une hypothèse $A$ de $\mathcal{P}(\Omega)$; autrement dit,
c’est la croyance affectée à A et aux hypothèses dont l'intersection
avec $A$ n’est pas vide. On la note $Pl(A)$. La valeur de la plausibilité est
toujours supérieure ou égale à celle de la croyance. Identiquement à la
croyance, $Pl(\varnothing)$ est nulle, et $Pl(\Omega)$ vaut $1$.

Formellement :
\begin{equation}
Pl(\varnothing) = 0
\end{equation}
\begin{equation}
Pl(\Omega) = 1
\end{equation}
\begin{equation}
Pl(A) = \sum_{B \slash B \cap A \neq \varnothing} m(B)
\end{equation}
La plausibilité peut aussi être dérivée à partir de la croyance:
\begin{equation}
Pl(A) = 1 - Bel(\bar{A}),
\end{equation}
où $A$ est le complément de $A$ dans $\Omega$.

La probabilité classique d’un événement $A$, $P(A)$ est représentée par
l’intervalle qui a comme bornes inférieure et supérieure $Bel(A)$ et $Pl(A)$,
respectivement, c’est-à-dire $$Pl(A) \leq P(A) \leq Bel(A).$$
Si $$Bel(A) = Pl(A),$$ alors la probabilité de $A$ est unique et $$P(A) = Bel(A) = Pl(A)$$

\subsubsection{Exemple}
On calcule les plausibilités pour le même exemple.

\begin{table}[h!]
\centering
\begin{tabular}{|l|l|}
\hline
Hypothèse $A$ & Plausibilité $Pl(A)$\\
\hline
$\varnothing$ & $0$ \\
\hline
$\{P1\}$ & $0.3 + 0.1 + 0.1 = 0.5$ \\
\hline
$\{P2\}$ & $0.5 + 0.1 = 0.6$ \\
\hline
$\{P3\}$ & $0.5 + 0.1 + 0.1 = 0.6$ \\
\hline
$\{P1, P2\}$ & $0.3 + 0.5 + 0.1 + 0.1 = 1$ \\
\hline
$\{P1, P3\}$ & $0.3 + 0.5 + 0.1 + 0.1 = 1$ \\
\hline
$\{P2, P3\}$ & $0.5 + 0.1 + 0.1 = 0.7$ \\
\hline
$\Omega$ & $1$ \\
\hline
\end{tabular}
\caption{Les plausibilités calculées à partir de la distribution de masses}
\end{table}
\section{Règles de combinaison}

Souvent, on obtient l’information à partir de sources distinctes et indépendantes,
et généralement ces sources ne sont pas parfaites, c’est pourquoi on a besoin de
fusionner les évidences fournies par ces sources en utilisant des règles de
combinaison. Il existe trois types de combinaison : \emph{conjonctive},
\emph{disjonctive}, et \emph{mixte}.

\subsection{Combinaison conjonctive}

Dempster a introduit une règle connue sous le nom de règle de combinaison
de Dempster (\emph{Dempster's Combination Rule}) pour fusionner deux pièces
d’évidence qui peuvent être représentées par deux fonctions de croyance.Cette
règle respecte les quatre propriétés algébriques suivantes : l’associativité,
la commutativité, l’idempotence et la continuité.

Soient deux fonctions de masse $m_1(A)$ et $m_2(A)$ associées à deux fonctions
de croyance $Bel_1(A)$ et $Bel_2(A)$ respectivement. $(Bel_1 \oplus Bel_2)(A)$
est définie à travers la masse $m_{1 \oplus 2}(A)$ :
\begin{equation}
m_{1 \oplus 2}(A) = \sum_{B \cap C = A} m_1(B) m_2(C).
\end{equation}

Néanmoins, cette règle a été critiquée par de nombreux chercheurs, parce qu’elle
n’est pas normalisée. En d’autre termes, elle permet d’affecter des masses à
des ensembles vides. Ce problème est dû au fait que l’intersection de deux
éléments focaux peut générer $\varnothing$, ce qui dénote un conflit entre les croyances.

Pour corriger ce problème, Shafer a proposé d’ignorer carrément les conflits
et d'étendre la distribution des masses. Le facteur de conflit est une valeur
$K < 1$ et définie par :
\begin{equation}
K = \sum_{B \cap C = \varnothing} m_1(B) m_2(C).
\end{equation}

Afin de normaliser la distribution de masse, la règle de Dempster doit être
multipliée par la quantité $(1 - K)^{-1}$. Donc la règle de Dempster-Shafer
est représentée par les équations suivantes :
\begin{equation}
m(\varnothing) = 0
\end{equation}
\begin{equation}
m(A) = \frac{1}{1-K} \sum_{B \cap C = A \neq \varnothing} m_1(B) m_2(C)
\end{equation}

Cependant, la règle de Shafer n’est pas toujours satisfaisante à cause de la
normalisation qui considère que le conflit vient de l’imperfection des sources.

D’autres chercheurs ont proposé des règles différents, comme Smets qui a introduit
une forme non normalisée pour prendre en compte les univers ouverts. Il suppose
que le conflit vient du fait que l’ensemble $\Omega$ n’est pas exhaustif. Pour
cette raison, la règle de Smets affecte une masse $K$ à $\varnothing$ qui représente
une hypothèse manquante. Elle est définie par ces équations :
\begin{equation}
K = m(\varnothing) = \sum_{B \cap C = \varnothing} m_1(B) m_2(C)
\end{equation}
\begin{equation}
m(A) = \sum_{B \cap C = A \neq \varnothing} m_1(B) m_2(C)
\end{equation}

Par contre, Yager propose un modèle d’un univers fermé ($\Omega$ est exhaustif).
La mesure de conflit est ajoutée à la mesure du cadre de discernement $\Omega$.
Donc le conflit sera transformé en ignorance. On obtient ces équations :
\begin{equation}
m(\varnothing) = \sum_{B \cap C = \varnothing} m_1(B) m_2(C)
\end{equation}
\begin{equation}
m(\Omega) = \left(1 - \sum_{A \in \mathcal{P}(\Omega)}
\sum_{B \cap C = A \neq \varnothing} m_1(B) m_2(C)\right) +
\sum_{B \cap C = \varnothing} m_1(B) m_2(C)
\end{equation}

\subsection{Combinaison disjonctive}

Dans la combinaison disjonctive on considère l’utilisation de l’union à la place
de l’intersection. Les éléments focaux sont obtenus à partir de la table d’union.
La fonction de masse de la règle de combinaison disjonctive est formalisée par l'équation :
\begin{equation}
m(A) = \sum_{B \cup C = A \neq \varnothing} m_1(B) m_2(C).
\end{equation}

Comme l’union de deux éléments focaux ne peut être jamais vide, on est sûr
que $m(\varnothing)=0$. Cela veut dire qu’il est impossible d’obtenir un conflit
lors de la combinaison. En contrepartie, l'union peut causer une perte de spécificité
vu que les masses sont élargies. Cette approche apparaît plus intéressante en
l’absence des informations nécessaires sur la fiabilité des sources.

\subsection{Combinaison mixte}

Dubois et Prade ont proposé une combinaison mixte comme un compromis entre la
combinaison conjonctive et la combinaison disjonctive et comme un essai de conserver
les avantages des deux. De même que la combinaison de Yager, le modèle de Dubois
et Prade suppose que le conflit est dû à la non fiabilité des sources. La règle de
Dubois et Prade est définie comme suit :
\begin{equation}
m(A) = \sum_{B \cap C = A \neq \varnothing} m_1(B) m_2(C) +
\sum_{\substack{B \cup C = A \neq \varnothing \\ B \cap C = \varnothing}} m_1(B) m_2(C)
\end{equation}

\section{Décision}

Prendre une décision veut dire choisir une hypothèse élémentaire parmi les autres
par la maximisation d'un critère. La sélection est établie en observant
la croyance et la plausibilité de chaque hypothèse résultante. Il existe trois règles
de décision : le \emph{maximum de croyance}, le \emph{maximum de plausibilité} et le
\emph{maximum de probabilité pignistique}

\subsection{Maximum de croyance}

Dans cette règle on s'intéresse à trouver l'hypothèse élémentaire $\omega_i$ qui
a la croyance maximale. C'est-à-dire que $\omega_i$ est choisie si elle satisfait cette
équation :
\begin{equation}
Bel(\omega_i) = \max_{1 \leq j \leq n} Bel(\omega_j)
\end{equation}
Cette méthode est connue par son caractère très pessimiste.

\subsection{Maximum de plausibilité}

En opposition à la règle précédente, l'élément $d_i$ est sélectionné par le maximum
de plausibilité. En d'autres termes, on choisit le $d_i$ qui résoud cette équation :
\begin{equation}
Pl(\omega_i) = \max_{1 \leq j \leq n} Pl(\omega_j).
\end{equation}
Ainsi, cette méthode a un caractère très optimiste.

\subsection{Maximum de probabilité pignistique}

Cette règle est proposée par Smets et Kennes. Elle consiste à transformer la fonction
de masse $m(A)$ en une fonction de probabilité $BetP(\omega)$. Cette transformation
est appelée \emph{transformation pignistique} et définie par cette équation :
\begin{equation}
BetP(\omega) = \sum_{\substack{A \subseteq \Omega \\
\omega \in A}} \frac{1}{|A|}  \frac{m(A)}{1-m(\varnothing)}.
\end{equation}

Dans cette transformation, la masse de croyance $m(A)$ est distribuée uniformément
sur l’ensemble des éléments de $A$. Le critère est de prendre l'élément qui a le
maximum de probabilité pignistique. On peut voir cette méthode comme un compromis
entre les deux précédentes.

\phantomsection
\addcontentsline{toc}{section}{Conclusion}
\section*{Conclusion}

Après cette présentation de la théorie des fonctions de
croyances, des règles de combinaisons qui peuvent si appliquer,
et des règles de décisions compatibles, nous allons maintenant pouvoir introduire
des notions sur les théories de l'incertain.