\documentclass[a4paper,french,12pt]{report}
\usepackage[utf8]{inputenc} % Encodage du fichier
\usepackage[T1]{fontenc} % Encodage des fonts nécessaire pour le Latin
\usepackage[english,frenchb]{babel} % Pour changer la langue des mots générés et choisir la bonne mise en page
\usepackage{lmodern} % Le latin modèrne
\usepackage[top=2cm, bottom=2cm, left=3cm, right=2cm]{geometry} % Définir les marges de la page
\usepackage{amsmath} % Pour des fonctions mathématiques
\usepackage{amssymb} % Pour les symbols mathématiques
\usepackage[hidelinks]{hyperref} % Pour les liens
\usepackage{fancyhdr} % Pour le style de la page
\usepackage[font=it]{caption} % Rendre les titres des tableaux italiques
\usepackage{microtype}
\usepackage{graphicx} % Pour les images
\usepackage{subcaption} % Pour mettre plusieurs images sur la même ligne
\usepackage{float} % Pour empêcher le déplacement des tableaux et des figures.
\usepackage[french,longend,boxruled,algoruled,linesnumbered,algochapter]{algorithm2e}%pour les algorithmes
\newcommand\mycommfont{\footnotesize\ttfamily\textcolor{blue}}%pour colorer les commenataires en bleu
\usepackage{babelbib} % Pour changer la langue dans la bibliographie
\usepackage{amsthm} % Pour les exemples
\usepackage{xspace}

\graphicspath{ {pictures/} } % Spécifier le répertoire contenant les images  
\DisableLigatures[f]{encoding=*}
\renewcommand \thechapter{\Roman{chapter}} % Utiliser la numéros romans pour les chapitre
\AtBeginDocument{% Changer "Table"
  \renewcommand\tablename{\itshape Tableau}
  \renewcommand{\figurename}{\itshape Figure}
	% Renommer la table des matières
	\renewcommand{\contentsname}{Sommaire}
}


\date{}
% Style de l'entête et le pied de la page 
\setlength{\headheight}{16pt}
\pagestyle{fancyplain}
\lhead{} % Enlever la section
\rhead{\fancyplain{}{\footnotesize \itshape{\nouppercase{\leftmark}}}} % Titre du chapitre en miniscule avec taille 10
\cfoot{} % Déplacer le numéro de la page
\rfoot{\fancyplain{\thepage}{\thepage}} % à droite de la page

% Espace entre les lignes
\linespread{1.3}
% Nom de notre application
\newcommand{\appname}{\textbf{combinateur d'évidences}\xspace}
\newcommand{\appName}{Combinateur d'évidences\xspace}
% Nom de la grande interface
\newcommand{\platformename}{\textbf{plateforme d'outils SII}\xspace}
\newcommand{\platformeName}{Plateforme d'outils SII\xspace}

% Gérer les exemples
\theoremstyle{definition}
\newtheorem{exemple}{Exemple}

\begin{document}
\newpage
\begin{titlepage}
\begin{center}
\section*{\textit{Remerciements}}
\end{center}
\textit{
\subparagraph{}
Tout d’abord, louange à « Allah » qui nous a guide sur le droit chemin tout au long du travail et nous a inspiré les bons pas et les justes reflexes. Sans sa miséricorde, ce travail n’aura pas abouti.
\subparagraph{}
Nous tenons à exprimer notre profonde reconnaissance au Madame Hadja Faiza Khellaf-Haned pour son encadrement, ces nombreux conseils et son soutien constant tout au long de notre Mémoire. Nous la remercions chaleureusement d’avoir encadré ce travail de Licence, avec beaucoup de compétence, d’enthousiasme et de disponibilité.
\subparagraph{}
Nous tenons à exprimer notre gratitude au Madame Khaoula Boutouhami pour nous avoir fait profiter de son expérience de recherche dans le domaine. Ses conseils sur l’aspect théorique et pratique qui nous ont été utiles et permis de mener à bien cette thèse.
\subparagraph{}
Nous remercions Mesdames A.Mokhtari, A.Akli d’avoir accepté de faire partie de notre jury de Licence. Nous leur exprimons notre profonde gratitude.
\subparagraph{}
Nous remercions tous les chercheurs, enseignants et membres du personnel de notre département d’informatique pour, nous avoir aidé pendant ces trois années d’études.
\subparagraph{}
Nous tenons à remercier  également tous  nos  amis, qui nous ont aidés au cours des trois années de notre parcours.
\subparagraph{}
Enfin, nous remercierions jamais assez nos parents nos frères et soeurs pour le soutien sans faille et permanent durant ces années, qui l'accomplissement de ce mémoire n'aurait pas eu lieu dans ces délais. Ils étaient là pour nous.
Nous tenons à leur dédier ce mémoire. 
}
\thispagestyle{empty}
\end{titlepage}
\tableofcontents
\listoffigures
\listofalgorithms
\parskip=0.6em


\phantomsection
\addcontentsline{toc}{chapter}{Introduction générale}
\chapter*{Introduction générale}
Les problématiques liées au domaine de l'intelligence artificielle tournent souvent autour de la représentation
des connaissances à modeliser et du raisonnement.

Notre projet traite un sujet situé dans le domaine de l'intelligence artificielle. Il a comme but la
manipulation de connaissances incertaines.

Il présente deux aspects principales. Il s'agit d'abord de réaliser un logiciel complet qui implémente l'une des
théories de ce domaine connue sous le nom de \emph{la théorie de Dempster-Shafer}. Ensuite, il a pour objectif de
créer une plateforme permettant de regrouper plusieurs outils developpés en utilisant diverses theories de l'incertain
telles que la théorie de probabilités et la théorie des possibilités qui traitent les connaissances incertaines ainsi
que la théorie des sous ensembles flous pour traiter les connaissances imprécises. Cette plateforme est destinée à
faciliter l'utilisation de ces outils en offrant une interface commune pour les exploiter.

C'est un projet qui répond à des besoins rencontrés aux cours et aux séances de TP par les étudiants du Master 2 de la
spécialité SII \footnote{Systèmes Informatiques Intelligents} de notre département \footnote{Département informatique
de la faculté d'électronique et d'informatique de l'USTHB}. Il est nécessaire en effet de combler le manque de logiciels
pour pouvoir effectuer les grands calculs requis par l'application de la théorie de Dempster-Shafer et pour résoudre un
problème donné. De plus, il est conçu pour offrir une interface pour tout les outils que les étudiants ont l'habitude
d'utiliser.

Le mémoire est organisé comme suit:\\
Dans le premier chapitre nous expliquons les concepts de la théorie de Dempster-Shafer et les règles de combinaison
et de décision associées. Dans le deuxième chapitre, nous exposons les théories de l'incertain qui sont implémentées
par les outils exploités à travers la plateforme. Dans le troisième chapitre nous présentons quelques algorithmes que
nous avons conçu et implémenté dans notre logiciel. Le dernier chapitre est consacré à la présentation de ce que
nous avons réalisé.
\chapter{Théorie de Dempster and Shafer}

\phantomsection
\addcontentsline{toc}{section}{Introduction}
\section*{Introduction}

La théorie de Dempster-Shafer, également connu comme la théorie de l'évidence,
ou la théorie des fonctions de croyance, est une théorie mathématique du raisonnement
sur l’évidence et la plausibilité. Elle a été développée par Glenn Shafer (1976)
en basant sur des travaux antérieurs de Arthur Dempster (1968). La théorie a attirer
l’attention des chercheurs de l'intelligence artificielle au début des années 80.

Dans un espace discret finis, la théorie de Dempster-Shafer représente la généralisation
de la théorie des probabilités où les les probabilités sont assignés à des ensembles
contrairement aux singletons mutuellement exclusifs.

\section{Concepts de base}

\subsection{Quelques notations}

Dans cette section, Nous allons présenter les conventions de notation utilisées dans le reste de ce mémoire.

Soit $\Omega$ un univers (Un ensemble fini qui contient tout les propositions auxquels on s'intéresse),
il est appelé également le cadre de discernement. On note $\varnothing$ l’ensemble qui
ne contient aucun élément de $\Omega$. $\mathcal{P}(\Omega)$ représente l’ensemble contenant tous
les parties (les sous-ensembles) de $\Omega$. On note aussi les opérations binaires
sur les ensembles $\subset$, $\cup$ et $\cap$, l’inclusion, l’union et l’intersection,
respectivement. On appel hypothèse un élément de $\mathcal{P}(\Omega)$.

La théorie de de Dempster-Shafer est caractérisée par trois fonctions principales : \textbf{l’assignement
de probabilité de base}, \textbf{la croyance} et \textbf{la plausibilité}.

\subsection{L’assignement de probabilité de base}

L’assignement de probabilité de base (\emph{Basic probability assignement bpa}),
appelé aussi la fonction de masse, noté $m$, est une fonction qui affecte le degré d’évidence disponible à
un élément $A$, et seulement à $A$, de $\mathcal{P}(\Omega)$ dans l’intervalle $[0,1]$, tel que
les $m(A)$ s’additionnent à $1$. $m(\varnothing)$ --qui représente l’absence de solution-- est
toujours égale à $0$ car $\Omega$ doit être exhaustive, et la somme de $m(A)$ vaut $1$. Chaque
élément A tel que $m(A) > 0$ est appelé un élément focal.

Généralement, le bpa n’est pas un équivalent de la fonction de probabilités classique. En effet,
la valeur exacte de la probabilité (dans le sens classique) appartient à un intervalle borné par
deux valeurs qui sont \emph{la croyance} et \emph{la plausibilité}.\\
Formellement, on peut représenter ça par :
\begin{equation}
m : \mathcal{P}(\Omega) \mapsto [0,1]
\end{equation}
\begin{equation}
m(\varnothing) = 0
\end{equation}
\begin{equation} \label{somme_masses}
\sum_{A \in \mathcal{P}(\Omega)} m(A) = 1
\end{equation}

\subsubsection{Exemple}
Dans une enquête d’un incident de vol, trois personnes \textit{P1}, \textit{P2}, \textit{P3}
sont accusés. L’enquêteur pense à 30\% que P1 est le voleur, à 50\% que l’un de P2 ou P3 soit
le voleur, et à 10\% que l’un de P1 ou P3 le soit.\\
Le cadre de discernement $\Omega = \{P1, P2, P3\}$.\\
$\mathcal{P}(\Omega) = \{\varnothing, \{P1\}, \{P2\}, \{P3\}, \{P1, P2\}, \{P1, P3\},
\{P2, P3\}, \Omega\}$.\\
La distribution des masses : $m(\{P1\}) = 0.3$, $m(\{P2, P3\})  = 0.5$,
$m(\{P1, P3\}) = 0.1$ et $m(\Omega) = 1 - (0.3+0.5+0.1) = 0.1$ à partir de l’équation
\ref{somme_masses}. Les masses des hypothèses restantes sont tout égaux à $0$.

\subsection{La croyance}

La croyance (Belief) représente la quantité du confiance qui supporte une hypothèse $A$ de 
$\mathcal{P}(\Omega)$ y compris tout ses parties. On la note $Cr(A$) ou plus souvent $Bel(A)$. Il est
claire que $Bel(\varnothing)$ est nul, et $Bel(\Omega)$ est toujours 1. Noter que la croyance est une
mesure non additive, c’est à dire que $Bel(A) + Bel(\Omega - A) \neq 1$.\\
Formellement :
\begin{equation}
Bel(\varnothing)=0
\end{equation}
\begin{equation}
Bel(\Omega)=1
\end{equation}
\begin{equation}
Bel(A) = \sum_{B \slash B \subseteq A} m(B)
\end{equation}
\begin{equation}
\forall p,q \in \mathcal{P}(\Omega) \medskip Bel(p \cup q) \geq Bel(p) + Bel(q) - Bel(p \cap q)
\end{equation}
Comme on peut obtenir le bpa avec la fonction inverse suivante :
\begin{equation}
m(A) = \sum_{B \slash B \subseteq A} (-1)^{|A-B|} Bel(B)
\end{equation}
Où $|A-B|$ est la différence des cardinalités entre les deux ensembles.

\subsubsection{Exemple}
Continuant de l'exemple précédent, on calcule la croyance de chaque hypothèse :

\begin{table}[ht]
\begin{center}
\begin{tabular}{|l|l|}
\hline
Hypothèse $A$ & Croyance $Bel(A)$\\
\hline
$\varnothing$ & $0$ \\
\hline
$\{P1\}$ & $0.3$ \\
\hline
$\{P2\}$ & $0$ \\
\hline
$\{P3\}$ & $0$ \\
\hline
$\{P1, P2\}$ & $0.3 + 0 + 0 = 0.3$ \\
\hline
$\{P1, P3\}$ & $0.3 + 0 + 0.1 = 0.4$ \\
\hline
$\{P2, P3\}$ & $0 + 0 + 0.1 = 0.1$ \\
\hline
$\Omega$ & $1$ \\
\hline
\end{tabular}
\caption{Les croyances calculées à partir de la distribution de masses}
\end{center}
\end{table}

\subsection{La plausibilité}

La plausibilité (Plausibility) représente la croyance qu’elle soit possible
d’être affectée à une hypothèse $A$ de $\mathcal{P}(\Omega)$, autrement dit
: c’est la croyance affectée à A et aux hypothèses dont leurs intersection
avec $A$ n’est pas vide. On la note $Pl(A)$. La valeur de la plausibilité est
toujours supérieur ou égale à celle de la croyance. Identiquement à la
croyance, $Pl(\varnothing)$ est nul, et $Pl(\Omega)$ vaut $1$.\\
Formellement :
\begin{equation}
Pl(\varnothing) = 0
\end{equation}
\begin{equation}
Pl(\Omega) = 1
\end{equation}
\begin{equation}
Pl(A) = \sum_{B \slash B \cap A \neq \varnothing} m(B)
\end{equation}
La plausibilité peut aussi être dérivée à partir de la croyance:
\begin{equation}
Pl(A) = 1 - Bel(\bar{A})
\end{equation}
Où $A$ est le complément de $A$ dans $\Omega$

La probabilité classique d’un évènement $A$, $P(A)$ est représentée par
l’intervalle qui a comme borne inférieur et supérieur $Bel(A)$ et $Pl(A)$,
respectivement, c’est à dire $Pl(A) \leq P(A) \leq Bel(A)$. Si $Bel(A) = Pl(A)$
alors la probabilité de $A$ est unique et $P(A) = Bel(A) = Pl(A)$.

\subsubsection{Exemple}
On calcule les plausibilités pour le même exemple:

\begin{table}[ht]
\begin{center}
\begin{tabular}{|l|l|}
\hline
Hypothèse $A$ & Plausibilité $Pl(A)$\\
\hline
$\varnothing$ & $0$ \\
\hline
$\{P1\}$ & $0.3 + 0.1 + 0.1 = 0.5$ \\
\hline
$\{P2\}$ & $0.5 + 0.1 = 0.6$ \\
\hline
$\{P3\}$ & $0.5 + 0.1 + 0.1 = 0.6$ \\
\hline
$\{P1, P2\}$ & $0.3 + 0.5 + 0.1 + 0.1 = 1$ \\
\hline
$\{P1, P3\}$ & $0.3 + 0.5 + 0.1 + 0.1 = 1$ \\
\hline
$\{P2, P3\}$ & $0.5 + 0.1 + 0.1 = 0.7$ \\
\hline
$\Omega$ & $1$ \\
\hline
\end{tabular}
\caption{Les plausibilités calculées à partir de la distribution de masses}
\end{center}
\end{table}
\section{Règles de combinaison}

Souvent, on obtient l’information à partir des sources distincts et indépendantes,
et ces sources ne sont pas parfaites généralement, c’est pour ça on a besoin de
fusionner les évidences fournies par ces sources en utilisant des règles de
combinaison. Il existe trois types de combinaison : \emph{conjonctive},
\emph{disjonctive}, et \emph{mixte}.

\subsection{Combinaison conjonctive}

Dempster à introduit une règle connue sous le nom de la règle de combinaison
de Dempster (\emph{Dempster's Combination Rule}) pour fusionner deux pièces
d’évidence qui peuvent être représentées par deux fonctions de croyance.Cette
règle respecte les quatre propriétés algébriques suivantes : l’associativité,
la commutativité, l’idempotence et la continuité.

Soient deux fonctions de masse $m_1(A)$ et $m_2(A)$ associés à deux fonctions
de croyances $Bel_1(A)$ et $Bel_2(A)$ respectivement, $(Bel_1 \oplus Bel_2)(A)$
est définie à travers la masse $m_{1 \oplus 2}(A)$ comme suit :
\begin{equation}
m_{1 \oplus 2}(A) = \sum_{B \cap C = A} m_1(B) m_2(C)
\end{equation}

Néanmoins, cette règle a été critiqué par de nombreux chercheurs, parce qu’elle
n’est pas normalisée, en d’autre termes, elle permet d’affecter des masses à
des ensembles vides, ce problème est dû au fait que l’intersection de deux
éléments focaux peut générer $\varnothing$, cela dénote le conflit entre les croyances.

Pour corriger ce problème, Shafer a proposé d’ignorer carrément les conflits
et étendre la distribution des masses. Le facteur de conflit est une valeur
$K < 1$ et définie par :
\begin{equation}
K = \sum_{B \cap C = \varnothing} m_1(B) m_2(C)
\end{equation}
Afin de normaliser la distribution de masse la règle de Dempster doit être
multipliée par la quantité $(1 - K)^{-1}$. Donc la règle de Dempster-Shafer
est représentée par l’équation suivante :
\begin{equation}
m(\varnothing) = 0
\end{equation}
\begin{equation}
m(A) = \frac{1}{1-K} \sum_{B \cap C = A \neq \varnothing} m_1(B) m_2(C)
\end{equation}
Cependant, la règle de Shafer n’est pas toujours satisfaisante à cause de la
normalisation qui considère que le conflit vient de l’imperfection des sources.

D’autres chercheurs ont proposé des règles différents, comme Smets qui a introduit
une forme non normalisée pour prendre en compte les univers ouverts. Smets suppose
que le conflit vient du fait que l’ensemble $\Omega$ n’est pas exhaustif. Pour
cette raison, la règle de Smets affecte une masse $K$ à $\varnothing$ qui représente
une hypothèse manquante. Elle est définie par ces équations :
\begin{equation}
K = m(\varnothing) = \sum_{B \cap C = \varnothing} m_1(B) m_2(C)
\end{equation}
\begin{equation}
m(A) = \sum_{B \cap C = A \neq \varnothing} m_1(B) m_2(C)
\end{equation}

Par contre, Yager propose un modèle d’un univers fermé ($\Omega$ est exhaustif).
La mesure de conflit est ajouté à la mesure du cadre de discernement $\Omega$.
Donc le conflit sera transformé en ignorance. On obtient ces équations :
\begin{equation}
m(\varnothing) = \sum_{B \cap C = \varnothing} m_1(B) m_2(C)
\end{equation}
\begin{equation}
m(\Omega) = \left(1 - \sum_{A \in \mathcal{P}(\Omega)}
\sum_{B \cap C = A \neq \varnothing} m_1(B) m_2(C)\right) +
\sum_{B \cap C = \varnothing} m_1(B) m_2(C)
\end{equation}

\subsection{Combinaison disjonctive}

Dans la combinaison disjonctive on considère l’utilisation de l’union à la place
de l’intersection. Les éléments focaux sont obtenus à partir de la table d’union.
La fonction de masse de la règle de combinaison disjonctive est formalisé par
cette équation :
\begin{equation}
m(A) = \sum_{B \cup C = A \neq \varnothing} m_1(B) m_2(C)
\end{equation}
Comme l’union de deux éléments focaux ne peut être jamais vide alors on est sûr
que $m(\varnothing)=0$. Cela veut dire qu’il est impossible d’obtenir un conflit
lors de combinaison. En contrepartie, elle peut causer une perte de spécificité
vu que les masses sont élargis. Cette approche apparaît plus intéressante à
l’absence des informations nécessaires sur la fiabilité des sources.

\subsection{Combinaison mixte}

Dubois et Prade ont proposé une combinaison mixte comme un compromis entre la
combinaison conjonctive et la combinaison disjonctive comme un essai de conserver
les avantages de ces deux. De même que la combinaison de  Yager, le modèle de Dubois
et Prade suppose que le conflit est dû à la non fiabilité des sources. La règle de
Dubois et Prade est définie comme suit :
\begin{equation}
m(A) = \sum_{B \cap C = A \neq \varnothing} m_1(B) m_2(C) +
\sum_{\substack{B \cup C = A \neq \varnothing \\ B \cap C = \varnothing}} m_1(B) m_2(C)
\end{equation}

\section{Décision}

Prendre une décision veut dire choisir une hypothèse élémentaire parmi les autres
par la maximisation d'un critère. La sélection est établit en observant
la croyance et la plausibilité de chaque hypothèse résultante. Il existe trois règles
de décision : le \emph{Maximum de croyance}, le \emph{Maximum de plausibilité} et le
\emph{Maximum de probabilité pignistique}

\subsection{Maximum de croyance}

Dans cette règle on s'intéresse à trouver l'hypothèse élémentaire $\omega_i$ qui
a la croyance maximale. C'est à dire, $\omega_i$ est choisie si il satisfait cette
équation :
\begin{equation}
Bel(\omega_i) = \max_{1 \leq j \leq n} Bel(\omega_j)
\end{equation}
Cette méthode est connue par son caractère très pessimiste.

\subsection{Maximum de plausibilité}

En opposition à la règle précédente, l'élément $d_i$ est sélectionné par le maximum
de plausibilité. En d'autres termes, on choisie le $d_i$ qui résoud cette équation :
\begin{equation}
Pl(\omega_i) = \max_{1 \leq j \leq n} Pl(\omega_j)
\end{equation}
Ainsi, cette méthode a un caractère très optimiste.

\subsection{Maximum de probabilité pignistique}

Cette règle est proposée par Smets et Kennes. Elle consiste à transformer la fonction
de masse $m(A)$ en une fonction de probabilité $BetP(\omega)$. Cette transformation
est appelé la \emph{transformation pignistique} et définie par cette équation :
\begin{equation}
BetP(\omega) = \sum_{\substack{A \subseteq \Omega \\
\omega \in A}} \frac{1}{|A|}  \frac{m(A)}{1-m(\varnothing)}
\end{equation}
Dans cette transformation la masse de croyance $m(A)$ est distribuée uniformément
sur l’ensemble des éléments de $A$. Le critère est de prendre l'élément qui a le
maximum de probabilité pignistique. On peut voir cette méthode comme un compromis
entre les deux précédentes.

\phantomsection
\addcontentsline{toc}{section}{Conclusion}
\section*{Conclusion}

Dans ce chapitre, nous avons fait une présentation de la théorie des fonctions de
croyances, les règles de combinaisons qui peuvent s'appliquer sur cette théorie,
et les règles de décisions compatibles. Dans le chapitre suivant nous allons introduire
des notions sur les théories de l'incertain.
\chapter{Théories de l'incertain}

\phantomsection
\addcontentsline{toc}{section}{Introduction}
\section*{Introduction}
Faisant suite aux travaux de Dempster et Shafer sur la théorie, différentes approches numériques ont été
proposées pour représenter les connaissances incertaines et imprécises. Nous allons nous focaliser, dans ce chapitre, sur ces certain nombre de théories modélisant les notions d’imperfection telles que l’incertain et l’imprécision.

\section{L’incertain}

Les approches logiques de l’incertain permettent d’utiliser un langage formel pour la description des connaissances et le raisonnement automatique. Elles constituent une référence aux autres formalismes surtout pour le raisonnement. Néanmoins, les connaissances ne  sont pas structurées.

Les principales représentations des connaissances incertaines sont le mode logique et le mode graphique\cite{hkhallafiThesis}.
Sur le plan de la représentation, le mode graphique est le plus explicite en soumettant les relations de dépendance qui existent entre les différentes variables. Sur le plan de raisonnement, le mode logique offre une machinerie d’inférence efficace.

Plusieurs chercheurs ont permis l’émergence d’un certain nombre de modèles graphiques offrant un cadre de représentation plus structuré.

La théorie des possibilités est une théorie de l’incertain ayant pour vocation de manipuler des connaissances incomplètes. Elle diffère de la théorie de probabilité  vu qu'elle manipule deux mesures duales : possibilité et nécessité. Cette théorie a été développée dans deux directions: la qualitative et la quantitative ce qui permet en fait de définir deux types de réseaux  possibilistes : les réseaux possibilistes basés sur le minimum (qualitatifs) et les réseau basés sur le produit (quantitatifs)\cite{hkhallafiThesis}\cite{kZebouchi2Thesis}.
\subsection{Théorie de probabilité}

Les réseaux bayésien peut être considérés comme une fusion de diagrammes d'incidence et le théorème de Bayes. La probabilité qu'un événement se produise étant donné que un autre événement a déjà eu lieu est appelé une probabilité conditionnelle. 

Un réseau bayésien est une représentation probabiliste des relations incertaines, qui s'est avéré être utile pour la modélisation de problèmes du monde réel. Le modèle probabiliste est décrit qualitativement par un graphe acyclique orienté, ou DAG (Directed Acyclic Graph). Les sommets du graphe représentent des variables, les arcs représentent la dépendance entre les variables. Le réseau comprennent aussi un ensemble de tables de probabilités, en indiquant les probabilités pour les vrais / fausses valeurs des variables.

Les avantages du  modèle graphique est qu’un réseau bayésien peut être utilisé pour apprendre les relations causales, et peut donc être utilisé pour obtenir la compréhension d'un domaine de problème et de prévoir les conséquences de l'intervention.

Afin de maintenir l'exploitation des réseaux bayésienne, Murphy a développe une boit a outil appelé le Bayes Net Boîte à outils (BNT).

Une des plus grandes forces de BNT \cite{BNT} est qu'elle offre une variété d'algorithmes d'inférence.


\subsection{Théorie de possibilité quantitative}

La théorie des possibilités offre deux modes de représentations.

\subsubsection{a- Mode graphique}

Un graphe possibiliste basé sur le produit (quantitatif), noté par $GP_{P}$, est un graphe possibiliste où les possibilités conditionnelles sont obtenues par un conditionnement de type produit. La distribution de possibilité des réseaux possibilistes bases sur le produit, notée par $\pi_{P}$, est obtenue par la règle de chainage 
\begin{equation}
\pi_{P} (V_1, \dots , V_N) = PROD_{i=1 \dots N} \prod  (V_i/PAR_{Vi})
\end{equation}
Ou PROD est l’opérateur produit\cite{BoBrDu2008.1}.
La boîte à outils PNT offre  différents algorithmes pour les réseaux causaux possibilistes basée sur le produit à connexions multiples et pour les polyarbres. 
\subsubsection{b- Mode logique}
La logique possibiliste offre un cadre général pour représenter les connaissances incertaines, en termes de formules logiques classiques auxquelles sont associées des pondérations appartenant à une échèle linaire de 0 à 1. Toute base possibiliste quantitative peut être codée de façon équivalente par un réseau causal possibiliste basé sur le produit.

–Il existe un programme de passage d’une base possibiliste quantitative vers un graphe causal basé sur le produit dans lequel l’affectation de l’ordre arbitraire entre les variables influe sur la structure du graphe. Il serait donc intéressant de définir des heuristiques afin d’obtenir la structure de graphe la moins complexe possible afin d’optimiser
le temps de la propagation.

–Il existe aussi un programme d’inférence appliqué à une base possibiliste quantitative. Il permet de transformer la base de connaissances afin de pouvoir utiliser un prouveur de la classe WMAXSAT connu pour être un des problèmes NP-difficiles\cite{hkhallafiThesis}.

\subsection{Théorie de possibilité qualitative}
\subsubsection{Mode graphique}

Un graphe possibiliste basé sur le minimum, noté par $GP_{M}$, est un graphe où les possibilités conditionnelles sont obtenues par le conditionnement minimum. La distribution de possibilité des réseaux possibilistes basée sur le minimum, notée par $\pi_{M}$, est obtenue par la règle de chainage :
\begin{equation}
 \pi_{M} (A_1, \dots, A_N) = MIN_{i=1 \dots N} \pi (A_i/teta A_i) 
\end{equation}
Ou MIN est  l’opérateur minimum\cite{BoBrDu2008.1}.

La boite à outils PNT possède aussi différents algorithmes pour les réseaux causaux possibilistes basés sur le minimum à connexions multiples et pour les polyarbres. 

\subsection{Décision dans l’incertain}
La théorie de la décision permet de modéliser le comportement d'un agent face
à des situations de choix en se basant sur un certain nombre d'axiomes. Le critère
de l'utilitée espérée est un critère dominant de cette approche, qui s'appuie sur une
base axiomatique très solide\cite{hkhaoulaThesis}.
\subsubsection{a- Mode graphique}
L’utilisation de modèles graphiques dans de nombreux problèmes de décision apporte de
l’expressivité et de l’efficacité de calcul tant pour la représentation des incertitudes que pour celle
des préférences. La structure du graphe exprime les spécificités du problème traité et est utilisée
pour propager des informations et optimiser des décisions. Plusieurs modèles graphiques peuvent
être utilisés pour le problème de la prise de décision séquentielle. On peut citer à titre d’exemple les réseaux possibilistes.

GraphViz02 est un programme qui permet de calculer la meilleure décision qualitative basée sur la représentation graphique des réseaux possibilistes. Pour faire ce calcul, le programme fait plusieurs démarches de fusion de réseau possibiliste codifiant les connaissances, avec le réseau possibiliste codifiant les préférences.Puis il génère de l’arbre de jonction correspondant\cite{hkhaoulaThesis}.
\subsubsection{b- Mode logique}
En plus des représentation graphiques, comme dans la logique possibiliste, il existe d’autres moyens de
représentation. Nous allons voir ici une représentation logique.

Les informations sont représentées en logique possibiliste au travers des formules quantifiées par des mesures de nécessité (qui sont définies par dualité par rapport aux mesures de possibilité), afin d'être exploitées pour la représentation de problèmes de décision sous incertitude, ainsi que pour le calcul de décision optimale.

Le programme DecPos s'intéresse aux problèmes de décision dans le cas multi-sources, dans lesquels il doit passer par un processus décisionnel permettant de fusionner les informations incertaines, afin de calculer la décision optimale dans les deux cas optimiste et pessimiste \cite{Noughithese}.

\section{L’imprécision}

La logique floue permet de solutionner tous les problèmes où nous disposons de connaissances imprécises, soumises à des incertitudes de nature non probabiliste.

cette dernière est une forme de logique polyvalente qui traite l’approximation, plutôt que le raisonnement fixe et exact. Par rapport à la logique binaire traditionnelle (où les variables peuvent prendre des valeurs vraies ou fausses), les variables de logique floue peuvent avoir une valeur de vérité qui varie en degré entre 0 et 1. La logique floue a été étendue pour gérer le concept de vérité partielle, où la valeur de vérité peut varier entre complètement vrai et complètement faux. 
\paragraph{Fuzzy logic} 
\vspace{1em}
La boîte à outils  Fuzzy est une collection de fonctions qui fournit des outils permettant de créer et d'éditer systèmes d'inférence floue. Et un bloc Simulink pour l'analyse, la conception et la simulation des systèmes basés sur la logique floue.

Elle permet de modéliser les comportements de systèmes complexes en utilisant des règles logiques simples, puis de mettre en œuvre ces règles dans un système d'inférence floue.

\section{Satisfiabilité}

Nous présentons ici l'outil UBCSAT en précisant en premier temps les notions sur les prouveurs SAT, MaxSAT et WMAXSAT (MAXSAT pondéré). 
\subsection{SAT}
Soit F une formule propositionnelle sous la forme normale conjonctive (CNF). Le problème SAT est un problème de décision NP-complet qui consiste à déterminer si F admet ou non un modèle\cite{hassenThesis}.

\textbf{Le problème de satisfiabilité propositionnelle (SAT)} est un sujet d'étude important dans de nombreux domaines de l'informatique, SAT est définie par les ces composantes :\\
\hspace{2em}Soit $X=\{x_1, x_2,\dots, x_n\}$, un ensemble de $n$ variables booléennes.\\
\hspace{2em}Soit $C=\{c_1, c_2,\dots, c_m\}$, un ensemble de $m$ clauses où :\\
\hspace{2em}chaque clause est une disjonction de littéraux,\\
\hspace{2em}chaque littéral est une variable ou sa négation.\\
\hspace{2em}Soit $D$, la donnée SAT, composé d'une conjonction de littéraux.

Le problème SAT consiste à déterminer s’il existe une assignation des variables $x_i$ de $X$ telle que la donnée $D$ soit satisfaisante. S’il existe une assignation de variables qui satisfait toutes les clauses, le SAT admet une réponse ‘Oui’ ou ‘Non sinon’\cite{hkhallafiThesis}.
\subsection{WMAXSAT}

toutefois, dans le cas de réponse négative du SAT, pour trouver toutes le nombre maximale de clauses pouvant être satisfaites à la fois. il a fallu définir \textbf{le problème du Maximum Satisfiabilité, ou MAXSAT}. Max-SAT est donc la version optimale du SAT dont le but est de satisfaire le nombre maximal de clauses.

Le problème avec le MAXSAT c’est qu’il associe à toutes les clauses le même poids, d’où la nécessité de définir \textbf{le problème du MAXSAT pondéré, ou WMAXSAT} cela permet d'attribuer des poids aux différentes clauses pour spécifier les clauses simultanément satisfaites en augmentant la somme de leur poids et en affaiblissant la somme des poids des clauses insatisfaites \cite{hkhallafiThesis}.


\textbf{UBCSAT (University of British Columbia SAT)}

L'un des défis de l'élaboration du projet UBCSAT était de construire un environnement flexible, riche en fonctionnalités sans compromettre l'efficacité algorithmique. Ce programme est performant sur les instances SAT issues de problèmes réels. Il offre la gestion des contraintes pseudo booléennes, et il décline un grand nombre de problèmes de décision ou d’optimisation en termes de problème SAT ou pseudo booléen    \cite{hassenThesis}.

Actuellement, UBCSAT comprend des implémentations de deux algorithmes conçus pour supporter les versions MAX-SAT, ainsi que MAX-SAT pondéré.

\phantomsection
\addcontentsline{toc}{section}{Conclusion}
\section*{Conclusion}
Les concepts de base de la théorie de l’incertain que nous venons d'introduire trouveront leurs affiliation dans les implémentations que nous avons intégrée dans notre réalisation. 
\parskip=0.6em
\chapter{Conception d'algorithmes}


\section*{Introduction}

Dans ce chapitre, nous allons détailler les différentes procédures et méthodes implémentée dans \textbf sur la théorie de Dempster-Shafer,décrite dans le premier chapitre.

les algorithmes qui suit, se déroule en un enchainement Précis,premièrement la procédure de préparation des masses qui fait des manipulations sur les masses sera expliquée dans la section de cette procédure, en suite la procédure d'appel de fusion et de croyance qui nécessite les donnée résultant de la procédure précédente qui fait appel a chaque procédure de fusion de croyance ,et à partir des données de la fusion les deux dernières procédures de calcule de croyance et de décision qui présente le résultat final de tous les algorithmes précédents .
\phantomsection
\addcontentsline{toc}{section}{Introduction}
\SetKwInput{KwIn}{Entrée}
\SetKwInput{KwOut}{Sortie}
\SetKw{KwTo}{à}
\SetKw{Begin}{Début}
\SetKw{End}{Fin}
\SetKw{KwRet}{retourne}
\SetKw{Retourner}{retourner}
\SetKwBlock{Début}{début}{Fin}
\SetKwComment{tcc}{/*}{*/}
\SetKwComment{tcp}{//}{}
\SetKwIF{Si}{SinonSi}{Sinon}{si}{alors}{sinon si}{sinon}{finsi}
\SetKwFor{Pour}{pour}{faire}{fin}
\SetKwFor{Tantque}{tantque}{faire}{fin}
\DontPrintSemicolon
\section{Preparation des masses}
Cette étape permet d'affecter à chaque agent une fiabilité par faire un affaiblissement a toutes les masses et l'attribuer à $\Omega$, et permet aussi de d'attribuer à $\Omega$ les masses non attribuées dans l'étape de collection d'information.   

ETATSDUMONDE est une variable utilisé dans l'algorithme, représente les états du monde collecté dans l'étape de collection d'information. \\
\begin{algorithm}[H]
\caption{Préparation des masses}
\BlankLine
\KwIn{
%\textit{$AGENTS$} = $\lbrace Agent_1, Agent_2\dots Agent_N \rbrace$,  \textit{$ETATSDUMONDE$} = $\lbrace Etat_1, Etat__2 \dots Etat__M \rbrace$}
\textit{$AGENTS$} = $\lbrace Agent_{1}, Agent_{2}\dots Agent_{M} \rbrace$,\\ \quad \quad \enspace \qquad \textit{$ETATSDUMONDE$} = $\lbrace Etat_{1}, Etat_{2}\dots Etat_{N} \rbrace$}
\KwOut{
 $\lbrace Agent_1, Agent_2\dots Agent_K \rbrace$}
\BlankLine 
\Begin

~~\\
$Ensemble \enspace SousEnsembles \gets SousEnsebles(ETATSDUMONDE)$ ~~\\
\tcc{La fonction SousEnsebles permet de générer tous les sous ensembles de l'ensemble donné en paramètre}

$Agents \enspace AgentsPreparés$
\\\Pour{$i \gets 0$ \KwTo $N$}{
$massSom \gets 0;$
\\\Si{$Agent(i).désactivé $}{
$ignorer$ \;
}
\Pour{$Chaque \enspace hypothèse \enspace de \enspace Agent$}{
\Si{$hypothèse \enspace \ne \enspace \Omega$}{
$Agent(i).Ajouter(hypothèse.id,hypothèse.masse \times Agent(i).Fiabilité) ;$
$ massSom \gets massSom + hypothèse.masse ;$ 
}
}

\Pour {$Chaque \enspace ensemble \enspace de \enspace SousEnsembles$}{
\Si{$ensemble \ne \varnothing \enspace \&\& \enspace Agent.hypothèse.Existe(ensemble)$}{
\Si{$ensemble = \Omega$}{
$Agent.Ajouter(ensemble.id,(1-massSom )\times Agent(i).Fiabilité) ;$
}
$Agent.Ajouter(ensemble.id,0);$
}

\Si{$ensemble$ $=$ $\Omega$}{
$Agent.Ajouter(\Omega.id,1-Fiabilité \times (\Omega.masse+ massSom);$ 
}
}
}
$AgentsPreparés.Ajouter(Agent);$
\\\Retourner{$AgentsPreparés$}

\End
\end{algorithm}


\section{Appel de procédures de fusion et de croyance}

Grâce à cet algorithme, nous pouvons appeler les méthodes de combinaison en passant les agents deux par deux en paramètres, de ce fait on peut fusionner un nombre plus de deux connaissances d'agents.\\

\begin{algorithm}[H]
\caption{Appel de procédures de fusion et de croyance}
\BlankLine
\KwIn{
$AGENT = \lbrace Agent_{1}, Agent_{2}\dots Agent_{N} \rbrace $}
\KwOut{$AGENTS$}
\BlankLine 
\Begin
\\
\Si{$N < 1 $}{
$Agent AgentTemporaire \gets AGENTS(1);$
\\
\Pour{$i \gets 2$ \KwTo $N$}{
  $cas (Methode) de $ \\
  $Dempster-Shafer : AgentTemporaire \gets MultiAgentDempsterShafer(AgentTemporaire,AGENTS(i));$\\
  $Dubois-Prade : AgentTemporaire \gets MultiAgentDuboisPrade(AgentTemporaire,AGENTS(i));$\\
  $Smets : AgentTemporaire \gets MultiAgentSmets(AgentTemporaire,AGENTS(i));$\\
  $Yager: AgentTemporaire \gets MultiAgentYager(AgentTemporaire,AGENTS(i));$
}
\vspace{1em}
\Si{$N =< 1 $}{
$CalculeCroyancePlausibilité(Agent);$
}
}
\Retourner{$AgentsPreparés$}
\end{algorithm}
\section{procédure de combinaison de d'information}
\begin{algorithm}[H]
\caption{Méthode de combinaison Dempster-Shafer}
\BlankLine
\KwIn{
$AGENT1 = \lbrace \lbrace hypothèse_{1},masse_{1} \rbrace \lbrace hypothèse_{2},masse_{2} \rbrace \dots \lbrace hypothèse_{n},masse_{n} \rbrace \rbrace $,\\ \quad \quad \enspace \qquad $AGENT2 = \lbrace \lbrace hypothèse_{1},masse_{1} \rbrace \lbrace hypothèse_{2},masse_{2} \rbrace \dots \lbrace hypothèse_{m},masse_{m} \rbrace \rbrace $}
\KwOut{$AGENTRes = \lbrace \lbrace hypothèse_{1},masse_{1} \rbrace \lbrace hypothèse_{2},masse_{2}  \rbrace \dots$ \\$ \lbrace hypothèse_{l},masse_{l} \rbrace \rbrace $}
\BlankLine 
\Begin
$AGENTRes.Ajouter(AGENT1,0)$
$AGENTRes.Ajouter(AGENT2,0)$
 \tcc{Ajouter touts les éléments de $AGENT1$ et $AGENT2$  avec une masse $= 0$ }
$ k \gets 0;$
\Pour{$i \gets 1$ \KwTo $N$}{
\Pour{$j \gets 1$ \KwTo $M$}{
\Si{$hypothèse_{i} \cap hypothèse_{j} = \varnothing $}{
$K \gets K + AGENT1.masse(i) \times AGENT2.masse(j);$
}
$AGENTRes.(hypothèse_{i} \cap hypothèse_{j}).masse \gets AGENTRes.(hypothèse_{i} \cap hypothèse_{j}).masse + hypothèse_{i}.masse \times hypothèse_{j}.masse ;$
}

\vspace{1em}
}
$k \gets 1-k;$ \\
\Pour{$k \gets 1$ \KwTo $L$}{
$AGENTRes(k).masse \gets AGENTRes(k).masse \times (1/k);$
}
\Retourner{$AGENTRes$}

\end{algorithm}
$\\ $Dans les algorithmes de combinaison qui suit une grande partie redondante n'est pas presenté car elle est déjà dans l'algorithme combinaison Dempster-Shafer, nous ne présenterons que les paries différentes.\\
\begin{algorithm}[H]
\caption{Méthode de combinaison Dubois-Prade}
\setcounter{AlgoLine}{3}
$AGENTRes.(hypothèse_{i} \cap hypothèse_{j}).masse \gets AGENTRes.(hypothèse_{i} \cap hypothèse_{j}).masse + hypothèse_{i}.masse \times hypothèse_{j}.masse ;$

$AGENTRes.(hypothèse_{i} \cup hypothèse_{j}).masse \gets AGENTRes.(hypothèse_{i} \cup hypothèse_{j}).masse + hypothèse_{i}.masse \times hypothèse_{j}.masse ;$
\end{algorithm}
\begin{algorithm}[H]
\setcounter{AlgoLine}{3}
\caption{Méthode de combinaison Smets}
\Si{$hypothèse_{i} \cap hypothèse_{j} = \varnothing $}{
$k \gets k + AGENTRes.(hypothèse_{i} \cap hypothèse_{j}).masse $}

$AGENTRes.(hypothèse_{i} \cap hypothèse_{j}).masse \gets AGENTRes.(hypothèse_{i} \cap hypothèse_{j}).masse + hypothèse_{i}.masse \times hypothèse_{j}.masse ;$
$AGENTRes.(\varnothing ).masse \gets k;$
\end{algorithm}
\begin{algorithm}[H]
\setcounter{AlgoLine}{9}
\caption{Méthode de combinaison Yager}
\Si{$LeDernierApelle()$}{
$AGENTRes.(\varnothing ).masse \gets AGENTRes.(\varnothing ).masse + k;$}
 \tcc{La fonction LeDernierApelle permet de vérifier si c'est le dernier appel de la Méthode de combinaison en comptant le nombre d'agents }
$AGENTRes.(hypothèse_{i} \cap hypothèse_{j}).masse \gets AGENTRes.(hypothèse_{i} \cap hypothèse_{j}).masse + hypothèse_{i}.masse \times hypothèse_{j}.masse ;$
$AGENTRes.(\varnothing ).masse \gets k;$
\end{algorithm}
\section{procédure de calcul de croyance}
\begin{algorithm}[H]
\caption{Calcule de Croyance et de Plausibilité}
\BlankLine
\KwIn{
$AGENTS = \lbrace \lbrace hypothèse_{1},masse_{1} \rbrace \lbrace hypothèse_{2},masse_{2} \rbrace \dots \lbrace hypothèse_{n},masse_{n} \rbrace \rbrace $}
\KwOut{$AGENTS = \lbrace \lbrace hypothèse_{1},masse_{1},CR_{1},PL_{1} \rbrace \lbrace hypothèse_{2},masse_{2},CR_{2},PL_{2}  \rbrace \dots$ \\$ \lbrace hypothèse_{n},masse_{n},CR_{n},PL_{n}  \rbrace \rbrace $}
\BlankLine 
\Begin

\Pour{$i \gets 1$ \KwTo $N$}{
$BL \gets 0;$
$PL \gets 0;$
\Pour{$j \gets 1$ \KwTo $N$}{
\Si{$hypothèse_{j} \subset hypothèse_{i} $}{
$BL \gets BL + ,masse_{i};$
}
}
\Pour{$j \gets 1$ \KwTo $N$}{
\Si{$hypothèse_{i} \cap hypothèse_{j} = \varnothing $}{
$PL \gets PL + ,masse_{j};$
}
}
\vspace{1em}

$AGENT(i).Ajouter(BL,PL);$

}
\Retourner{$AGENT$}
\end{algorithm}
\begin{algorithm}[H]
\caption{Méthode de calcul de décision}
\Pour{$i \gets 1$ \KwTo $N$}{

\end{algorithm}
%\phantomsection
%\addcontentsline{toc}{section}{Conclusion}
\section*{Conclusion}

\chapter{Réalisation de la toolbox}

\section{Introduction}

Dans ce chapitre, nous introduisons l'implémentation de notre projet. D'abord, nous 
décrirons les différents outils utilisés pour la réalisation de la toolbox. Ensuite,
nous possèderons à démontrer notre toolbox de Dempster-Shafer en détaillant ses
fonctionnalitées. Enfin, nous présentons la grande toolbox qui regroupe plusieurs
logiciels reliée aux théories de l'incertain.

\section{Les languages de programmations et les bibliothèques utilisés}

Nous avons programmé l'interface de la toolbox de Dempster-Shafer en \textbf{Python 3}
en utilisant la bibliothèque \textbf{PyQt4}. Nous avons également réalisé la grande
interface avec \textbf{Java Swing}.

\section{La toolbox de Dempster-Shafer}

Ce logiciel est constitué à partir de deux parties, le noyau et l'interface graphique.
Il peut fonctionner sur tout les systèmes d'exploitations majeurs comme \textbf{Windows},
\textbf{Mac~OS~X}, \textbf{GNU/Linux} et \textbf{BSD}.

Le noyau est responsable d'effectuer les calculs après la lecture d'un fichier XML
contenant les données nécessaires pour l'application de la théorie de Dempster-Shafer.
Les résultats seront écrits dans un autre fichier XML.

L'interface graphique est composée d'une barre de menu et de trois parties, les états
du monde, les hypothèses, et les agents. La barre contient les fonctionnalités principales
pour enregister, ouvrir, réinitialiser le projet et fermer l'application dans le menu
\textbf{Fichier}. Dans le menu \textbf{Projet}, on trouve les actions liés à la manipulation
du projet courant.
\phantomsection
\addcontentsline{toc}{chapter}{Conclusion générale}
\chapter*{Conclusion générale}

Sur le plan théorique, notre travail se situe dans le cadre de traitement des connaissances incertaines,
principalement, sur la fusion de croyances sous la théorie de Dempster-Shafer. En plus, il comporte d'autres
théories d'incertain.

Nous avons fait une étude pour comprendre et pouvoir exposer les notions générales de la théorie de Dempster-Shafer,
en expliquant les fonctions de croyance, les règles de combinaisons et les règles de décision. 
Nous avons aussi conçu et implémenté les algorithmes nécessaires pour l'exploitation complète de cette théorie. De plus,
nous avons présenté brièvement les autres théories d'incertain qui sont situées dans le même domaine de notre sujet.

Sur le plan pratique, nous avons réalisé deux applications. La première est un logiciel qui permet la représentation
et puis la résolution d'un problème d'incertitude dans un système multi-agents en appliquant la théorie de Dempster-Shafer.
La deuxième application est une plateforme qui fait appel à des différent outils qui implémentent des théories de
l'intélligence artificielle.

Comme perspectives, il serait intéressent de modifier notre plateforme en ajoutant des interfaces pour d'autres outils
manquantes dans cette version, ou même de supprimer les existantes s'ils ne seront plus utiles. Pour cela, nous avons créé
la plateforme d'une façon à faciliter sa extention.

Nous souhaitons que notre travail soit bénéfique pour notre université et nous espérons qu'il sera utilisé dans autres
établissements. Il peut fournir un aide important pour les gens spécialisés dans le domaine de l'intélligence artificielle.
\bibliographystyle{babplain}
\phantomsection
\addcontentsline{toc}{chapter}{Bibliographie}
\bibliography{bibliographie}
\end{document}          
