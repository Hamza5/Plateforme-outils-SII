\chapter{La réalisation du \appname et de la \platformename}
\phantomsection
\addcontentsline{toc}{section}{Introduction}
\section*{Introduction}

Dans ce chapitre, nous introduisons l'implémentation de notre projet. D'abord, nous
citons les différents langages programmation et les bibliothèques que nous avons manipulé.
Ensuite, nous présentons le \appname en détaillant ses fonctionnalités. Enfin, nous passons
à décrir la grande \platformename qui regroupe plusieurs programmes reliés aux théories de
l'incertain. Nous allons aussi expliquer comment cette platforme intéragit avec ces programmes.

\section{Les langages de programmation et les bibliothèques utilisées}

Nous avons programmé l'interface du \appname en \textbf{Python 3}
en utilisant la bibliothèque \textbf{PyQt4}. Nous avons également réalisé la \platformename
avec \textbf{Java Swing}. En plus, il y a plusieurs scriptes utilisés par cette
platforme sont écrits en \textbf{MATLAB}. D'autres outils sont programmés en language
\textbf{C} et compilés sous une forme exécutable. Il y a aussi quelques programmes
qui nécessite la présence de l'environement \textbf{Cygwin}.

\section{\appName}

Ce logiciel est constitué à partir de deux parties, le noyau et l'interface graphique.
Il peut fonctionner sur tous les systèmes d'exploitation majeurs comme \textbf{Windows},
\textbf{\mbox{Mac OS X}}, \textbf{\mbox{GNU/Linux}} et \textbf{BSD}.

Le noyau est responsable d'effectuer les calculs après la lecture d'un fichier XML
contenant les données nécessaires pour l'application de la théorie de Dempster-Shafer.
Les résultats seront écrits dans un autre fichier XML. L'utilisateur n'a pas besoin
de comprendre la structure de ces fichiers, il lui suffit d'utiliser l'interface
graphique.

Cette interface est composée d'une barre de menu et de trois parties, les états
du monde, les hypothèses, et les agents. La barre contient les fonctionnalités principales
pour enregister, ouvrir, réinitialiser le projet et fermer l'application dans le menu
\textbf{Fichier}. Dans le menu \textbf{Projet}, on trouve les actions liées à la manipulation
du projet courant. On peut y changer le titre et donner une description au projet.\\[1em]

\begin{figure}[H]
\begin{subfigure}{0.49\textwidth}
\includegraphics[width=\textwidth]{Inetrface_principale_menu_fichier}
\caption{Le menu \textbf{Fichier}}
\end{subfigure}
\hfill
\begin{subfigure}{0.49\textwidth}
\includegraphics[width=\textwidth]{Inetrface_principale_menu_projet}
\caption{Le menu \textbf{Projet}}
\end{subfigure}
\caption{L'interface principale du \appname}
\end{figure}

Les états du monde, les hypothèses et les agents doivent s'ajouter dans cet ordre à partir
de ce menu ou par un clic droit dans leurs champs. \`A partir de ce menu, on peut choisir la
méthode de fusion et la méthode de décision.

On commence par ajouter tous les états du monde. Ensuite, chaque fois qu'on sélectionne certains
états, on établit une hypothèse à partir de ces états.

\begin{figure}[H]
\begin{subfigure}{0.49\textwidth}
\includegraphics[width=\textwidth]{ajouter_etat}
\caption{Ajouter les états du monde}
\end{subfigure}
\hfill
\begin{subfigure}{0.49\textwidth}
\includegraphics[width=\textwidth]{ajouter_hypothese}
\caption{Ajouter une hypothèse à partir des états}
\end{subfigure}
\caption{L'ajout des états du monde et les hypothèses}
\end{figure}

Par la suite, on doit procéder à ajouter les agents. Chaque agent doit avoir un nom, un niveau de
fiabilité et des hypothèses tels que à chaque hypothèse est affectée à une masse.

\begin{figure}[H]
\begin{subfigure}{0.49\textwidth}
\includegraphics[width=\textwidth]{ajouter_agent}
\caption{Dialogue de l'ajout d'un agent}
\end{subfigure}
\hfill
\begin{subfigure}{0.49\textwidth}
\includegraphics[width=\textwidth]{affecter_masse}
\caption{Dialogue de l'affectation d'une hypothèse}
\end{subfigure}
\caption{L'ajout d'un agent et l'affectation de ses hypothèses}
\end{figure}

Avant de passer au calcul, on doit enregistrer ces données dans un fichier. Il faut aussi choisir un nom
et un emplacement pour le fichier. Il aura par défaut l'extention \texttt{.dsti.xml}. Si on ignore cette
étape le programme demandera de la faire avant qu'on puisse continuer.

\begin{figure}[H]
\centering
\includegraphics[width=0.8\textwidth]{Enregistrer}
\caption{Enregistrer les données}
\end{figure}

Enfin, on peut lancer le calcul. Cette étape risque de prendre un temps important si le nombre des états
du mondes et/ou des agents est assez grand. Pour cela, un dialogue d'attente est affiché pendant l'exécution
du noyau en arrière-plan. L'utilisateur peut annuler cette opération à tout moment.

Quand le calcul se termine, un dialogue contenant tout les résultats sera affiché. Les résultats sont obtenus
par la lecture du fichier généré par le noyau qui a le même chemin que le fichier sauvegardé, sauf qu'il porte
l'extention \texttt{.dsto.xml}.

\begin{figure}[H]
\begin{subfigure}{0.39\textwidth}
\includegraphics[width=\textwidth]{Dialogue_attente}
\caption{Dialogue d'attente}
\end{subfigure}
\hfill
\begin{subfigure}{0.59\textwidth}
\includegraphics[width=\textwidth]{Dialogue_resultats}
\caption{Dialogue des résultats}
\end{subfigure}
\caption{Le calcul et l'affichage des résultats}
\end{figure}

Plus tard, on peut ouvrir les fichiers de données en utilisant l'action \textbf{Ouvrir} à partir du menu fichier. Comme
on peut afficher les résultats en ouvrant le fichier correspondant en utilisant l'action \textbf{Ouvrir des résultats}.

\begin{figure}[H]
\begin{subfigure}{0.49\textwidth}
\includegraphics[width=\textwidth]{Ouvrir}
\caption{Dialogue d'ouverture d'un fichier de données}
\end{subfigure}
\hfill
\begin{subfigure}{0.49\textwidth}
\includegraphics[width=\textwidth]{Ouvrir_resultats}
\caption{Dialogue d'ouverture de résultats}
\end{subfigure}
\caption{Les dialogues de l'ouverture d'un fichier}
\end{figure}

\section{\platformeName}
Ce programme est une interface composée de plusieurs panneaux dans lequel chaque panneau est accessible
à partir de son titre dans la barre des onglets. Un panneau comporte une ou plusieurs interfaces pour
un ou plus d'un programme externe.  La \platformename est facilement extensible; pour ajouter un nouveau
paneau dans cette interface il suffit de créer une classe personnalisée qui dérive de la classe
\mbox{\texttt{javax.swing.JPanel}} dans le paquet \texttt{Plugins}.
