\chapter{Réalisation de la toolbox}

\section{Introduction}

Dans ce chapitre, nous introduisons l'implémentation de notre projet. D'abord, nous 
décrirons les différents outils utilisés pour la réalisation de la toolbox. Ensuite,
nous possèderons à démontrer notre toolbox de Dempster-Shafer en détaillant ses
fonctionnalitées. Enfin, nous présentons la grande toolbox qui regroupe plusieurs
logiciels reliée aux théories de l'incertain.

\section{Les languages de programmations et les bibliothèques utilisés}

Nous avons programmé l'interface de la toolbox de Dempster-Shafer en \textbf{Python 3}
en utilisant la bibliothèque \textbf{PyQt4}. Nous avons également réalisé la grande
interface avec \textbf{Java Swing}.

\section{La toolbox de Dempster-Shafer}

Ce logiciel est constituée à partir de deux parties, le noyau et l'interface graphique.

Le noyau est responsable d'effectuer les calculs après la lecture d'un fichier XML
contenant les données nécessaires pour l'application de la théorie de Dempster-Shafer.
Les résultats seront écrits dans un autre fichier XML.
