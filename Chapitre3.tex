\parskip=0.6em
\chapter{Conception d'algorithmes}


\section*{Introduction}

Dans ce chapitre, nous allons détailler les différentes procédures et méthodes implémentée dans \textbf sur la théorie de Dempster-Shafer,décrite dans le premier chapitre.

les algorithmes qui suit, se déroule en un enchainement Précis,premièrement la procédure de préparation des masses qui fait des manipulations sur les masses sera expliquée dans la section de cette procédure, en suite la procédure d'appel de fusion et de croyance qui nécessite les donnée résultant de la procédure précédente qui fait appel a chaque procédure de fusion de croyance ,et à partir des données de la fusion les deux dernières procédures de calcule de croyance et de décision qui présente le résultat final de tous les algorithmes précédents .
\phantomsection
\addcontentsline{toc}{section}{Introduction}
\SetKwInput{KwIn}{Entrée}
\SetKwInput{KwOut}{Sortie}
\SetKw{KwTo}{à}
\SetKw{Begin}{Début}
\SetKw{End}{Fin}
\SetKw{KwRet}{retourne}
\SetKw{Retourner}{retourner}
\SetKwBlock{Début}{début}{Fin}
\SetKwComment{tcc}{/*}{*/}
\SetKwComment{tcp}{//}{}
\SetKwIF{Si}{SinonSi}{Sinon}{si}{alors}{sinon si}{sinon}{finsi}
\SetKwFor{Pour}{pour}{faire}{fin}
\SetKwFor{Tantque}{tantque}{faire}{fin}
\DontPrintSemicolon
\section{Preparation des masses}
Cette étape permet d'affecter à chaque agent une fiabilité par faire un affaiblissement a toutes les masses et l'attribuer à $\Omega$, et permet aussi de d'attribuer à $\Omega$ les masses non attribuées dans l'étape de collection d'information.   

ETATSDUMONDE est une variable utilisé dans l'algorithme, représente les états du monde collecté dans l'étape de collection d'information. \\
\begin{algorithm}[H]
\caption{Préparation des masses}
\BlankLine
\KwIn{
%\textit{$AGENTS$} = $\lbrace Agent_1, Agent_2\dots Agent_N \rbrace$,  \textit{$ETATSDUMONDE$} = $\lbrace Etat_1, Etat__2 \dots Etat__M \rbrace$}
\textit{$AGENTS$} = $\lbrace Agent_{1}, Agent_{2}\dots Agent_{M} \rbrace$,\\ \quad \quad \enspace \qquad \textit{$ETATSDUMONDE$} = $\lbrace Etat_{1}, Etat_{2}\dots Etat_{N} \rbrace$}
\KwOut{
 $\lbrace Agent_1, Agent_2\dots Agent_K \rbrace$}
\BlankLine 
\Begin

~~\\
$Ensemble \enspace SousEnsembles \gets SousEnsebles(ETATSDUMONDE)$ ~~\\
\tcc{La fonction SousEnsebles permet de générer tous les sous ensembles de l'ensemble donné en paramètre}

$Agents \enspace AgentsPreparés$
\\\Pour{$i \gets 0$ \KwTo $N$}{
$massSom \gets 0;$
\\\Si{$Agent(i).désactivé $}{
$ignorer$ \;
}
\Pour{$Chaque \enspace hypothèse \enspace de \enspace Agent$}{
\Si{$hypothèse \enspace \ne \enspace \Omega$}{
$Agent(i).Ajouter(hypothèse.id,hypothèse.masse \times Agent(i).Fiabilité) ;$
$ massSom \gets massSom + hypothèse.masse ;$ 
}
}

\Pour {$Chaque \enspace ensemble \enspace de \enspace SousEnsembles$}{
\Si{$ensemble \ne \varnothing \enspace \&\& \enspace Agent.hypothèse.Existe(ensemble)$}{
\Si{$ensemble = \Omega$}{
$Agent.Ajouter(ensemble.id,(1-massSom )\times Agent(i).Fiabilité) ;$
}
$Agent.Ajouter(ensemble.id,0);$
}

\Si{$ensemble$ $=$ $\Omega$}{
$Agent.Ajouter(\Omega.id,1-Fiabilité \times (\Omega.masse+ massSom);$ 
}
}
}
$AgentsPreparés.Ajouter(Agent);$
\\\Retourner{$AgentsPreparés$}

\End
\end{algorithm}


\section{Appel de procédures de fusion et de croyance}

Grâce à cet algorithme, nous pouvons appeler les méthodes de combinaison en passant les agents deux par deux en paramètres, de ce fait on peut fusionner un nombre plus de deux connaissances d'agents.\\

\begin{algorithm}[H]
\caption{Appel de procédures de fusion et de croyance}
\BlankLine
\KwIn{
$AGENT = \lbrace Agent_{1}, Agent_{2}\dots Agent_{N} \rbrace $}
\KwOut{$AGENTS$}
\BlankLine 
\Begin
\\
\Si{$N < 1 $}{
$Agent AgentTemporaire \gets AGENTS(1);$
\\
\Pour{$i \gets 2$ \KwTo $N$}{
  $cas (Methode) de $ \\
  $Dempster-Shafer : AgentTemporaire \gets MultiAgentDempsterShafer(AgentTemporaire,AGENTS(i));$\\
  $Dubois-Prade : AgentTemporaire \gets MultiAgentDuboisPrade(AgentTemporaire,AGENTS(i));$\\
  $Smets : AgentTemporaire \gets MultiAgentSmets(AgentTemporaire,AGENTS(i));$\\
  $Yager: AgentTemporaire \gets MultiAgentYager(AgentTemporaire,AGENTS(i));$
}
\vspace{1em}
\Si{$N =< 1 $}{
$CalculeCroyancePlausibilité(Agent);$
}
}
\Retourner{$AgentsPreparés$}
\end{algorithm}
\section{procédure de combinaison de d'information}
\begin{algorithm}[H]
\caption{Méthode de combinaison Dempster-Shafer}
\BlankLine
\KwIn{
$AGENT1 = \lbrace \lbrace hypothèse_{1},masse_{1} \rbrace \lbrace hypothèse_{2},masse_{2} \rbrace \dots \lbrace hypothèse_{n},masse_{n} \rbrace \rbrace $,\\ \quad \quad \enspace \qquad $AGENT2 = \lbrace \lbrace hypothèse_{1},masse_{1} \rbrace \lbrace hypothèse_{2},masse_{2} \rbrace \dots \lbrace hypothèse_{m},masse_{m} \rbrace \rbrace $}
\KwOut{$AGENTRes = \lbrace \lbrace hypothèse_{1},masse_{1} \rbrace \lbrace hypothèse_{2},masse_{2}  \rbrace \dots$ \\$ \lbrace hypothèse_{l},masse_{l} \rbrace \rbrace $}
\BlankLine 
\Begin
$AGENTRes.Ajouter(AGENT1,0)$
$AGENTRes.Ajouter(AGENT2,0)$
 \tcc{Ajouter touts les éléments de $AGENT1$ et $AGENT2$  avec une masse $= 0$ }
$ k \gets 0;$
\Pour{$i \gets 1$ \KwTo $N$}{
\Pour{$j \gets 1$ \KwTo $M$}{
\Si{$hypothèse_{i} \cap hypothèse_{j} = \varnothing $}{
$K \gets K + AGENT1.masse(i) \times AGENT2.masse(j);$
}
$AGENTRes.(hypothèse_{i} \cap hypothèse_{j}).masse \gets AGENTRes.(hypothèse_{i} \cap hypothèse_{j}).masse + hypothèse_{i}.masse \times hypothèse_{j}.masse ;$
}

\vspace{1em}
}
$k \gets 1-k;$ \\
\Pour{$k \gets 1$ \KwTo $L$}{
$AGENTRes(k).masse \gets AGENTRes(k).masse \times (1/k);$
}
\Retourner{$AGENTRes$}

\end{algorithm}
$\\ $Dans les algorithmes de combinaison qui suit une grande partie redondante n'est pas presenté car elle est déjà dans l'algorithme combinaison Dempster-Shafer, nous ne présenterons que les paries différentes.\\
\begin{algorithm}[H]
\caption{Méthode de combinaison Dubois-Prade}
\setcounter{AlgoLine}{3}
$AGENTRes.(hypothèse_{i} \cap hypothèse_{j}).masse \gets AGENTRes.(hypothèse_{i} \cap hypothèse_{j}).masse + hypothèse_{i}.masse \times hypothèse_{j}.masse ;$

$AGENTRes.(hypothèse_{i} \cup hypothèse_{j}).masse \gets AGENTRes.(hypothèse_{i} \cup hypothèse_{j}).masse + hypothèse_{i}.masse \times hypothèse_{j}.masse ;$
\end{algorithm}
\begin{algorithm}[H]
\setcounter{AlgoLine}{3}
\caption{Méthode de combinaison Smets}
\Si{$hypothèse_{i} \cap hypothèse_{j} = \varnothing $}{
$k \gets k + AGENTRes.(hypothèse_{i} \cap hypothèse_{j}).masse $}

$AGENTRes.(hypothèse_{i} \cap hypothèse_{j}).masse \gets AGENTRes.(hypothèse_{i} \cap hypothèse_{j}).masse + hypothèse_{i}.masse \times hypothèse_{j}.masse ;$
$AGENTRes.(\varnothing ).masse \gets k;$
\end{algorithm}
\begin{algorithm}[H]
\setcounter{AlgoLine}{9}
\caption{Méthode de combinaison Yager}
\Si{$LeDernierApelle()$}{
$AGENTRes.(\varnothing ).masse \gets AGENTRes.(\varnothing ).masse + k;$}
 \tcc{La fonction LeDernierApelle permet de vérifier si c'est le dernier appel de la Méthode de combinaison en comptant le nombre d'agents }
$AGENTRes.(hypothèse_{i} \cap hypothèse_{j}).masse \gets AGENTRes.(hypothèse_{i} \cap hypothèse_{j}).masse + hypothèse_{i}.masse \times hypothèse_{j}.masse ;$
$AGENTRes.(\varnothing ).masse \gets k;$
\end{algorithm}
\section{procédure de calcul de croyance}
\begin{algorithm}[H]
\caption{Calcule de Croyance et de Plausibilité}
\BlankLine
\KwIn{
$AGENTS = \lbrace \lbrace hypothèse_{1},masse_{1} \rbrace \lbrace hypothèse_{2},masse_{2} \rbrace \dots \lbrace hypothèse_{n},masse_{n} \rbrace \rbrace $}
\KwOut{$AGENTS = \lbrace \lbrace hypothèse_{1},masse_{1},CR_{1},PL_{1} \rbrace \lbrace hypothèse_{2},masse_{2},CR_{2},PL_{2}  \rbrace \dots$ \\$ \lbrace hypothèse_{n},masse_{n},CR_{n},PL_{n}  \rbrace \rbrace $}
\BlankLine 
\Begin

\Pour{$i \gets 1$ \KwTo $N$}{
$BL \gets 0;$
$PL \gets 0;$
\Pour{$j \gets 1$ \KwTo $N$}{
\Si{$hypothèse_{j} \subset hypothèse_{i} $}{
$BL \gets BL + ,masse_{i};$
}
}
\Pour{$j \gets 1$ \KwTo $N$}{
\Si{$hypothèse_{i} \cap hypothèse_{j} = \varnothing $}{
$PL \gets PL + ,masse_{j};$
}
}
\vspace{1em}

$AGENT(i).Ajouter(BL,PL);$

}
\Retourner{$AGENT$}
\end{algorithm}
\begin{algorithm}[H]
\caption{Méthode de calcul de décision}
\Pour{$i \gets 1$ \KwTo $N$}{

\end{algorithm}
%\phantomsection
%\addcontentsline{toc}{section}{Conclusion}
\section*{Conclusion}
