\chapter{Conception des algorithmes}

\phantomsection
\addcontentsline{toc}{section}{Introduction}
\section*{Introduction}

Dans ce chapitre, nous allons détailler les différentes procédures et méthodes que nous avons implémentées dans \textbf sur la théorie de Dempster-Shafer, décrite dans le premier chapitre.

Les algorithmes qui suivent, se déroule en un enchainement précis.
Premièrement la procédure de préparation qui fait des manipulations sur les masses sera expliquée dans la section de cette procédure. Vient ensuite la procédure d'appel de fusion et de croyance qui nécessite les données résultant de la procédure précédente et qui fait appel à chaque procédure de fusion de croyance. Enfin à partir des données de la fusion viennent les deux dernières procédures de calcul de croyance et de décision qui présentent le résultat final de tous les algorithmes précédents.
\SetKwInput{KwIn}{Entrée}
\SetKwInput{KwOut}{Sortie}
\SetKw{KwTo}{à}
\SetKw{Begin}{Début}
\SetKw{End}{Fin}
\SetKw{KwRet}{retourne}
\SetKw{Retourner}{retourner}
\SetKwBlock{Début}{début}{Fin}
\SetKwComment{tcc}{/*}{*/}
\SetKwComment{tcp}{//}{}
%\SetKwIF{Si}{SinonSi}{Sinon}{si}{alors}{sinon si}{sinon}{finsi}
%\SetKwFor{Pour}{pour}{faire}{fin}
%\SetKwcaseOf{Switch}{switch}{faire}{fin}
\SetKwFor{Tantque}{tantque}{faire}{fin}
%\Suivant{condition}{bloc du Suivant-cas-alors} \uCas{cas où}{bloc de ce cas sans fin}
%\Cas{cas où}{bloc de ce cas}
%\lCas{cas où}{ligne de ce cas}
%\Autre{bloc de l’alternative}
\SetKwSwitch{Suivant}{Cas}{Autre}{suivant}{faire}{cas où}

\lAutre{ligne de l’alternative}
\DontPrintSemicolon
\section{Preparation des masses}
Cette étape permet d'affecter à chaque agent une fiabilité par faire un affaiblissement a toutes les masses et l'attribuer à $\Omega$, et permet aussi de d'attribuer à $\Omega$ les masses non attribuées dans l'étape de collection d'information.   

$EtatsDuMonde$ est une variable utilisé dans l'algorithme, représente les états du monde collecté dans l'étape de collection d'information. \\
\begin{algorithm}[H]
\caption{Préparation des masses}
\BlankLine
\KwIn{
%\textit{$AGENTS$} = $\lbrace Agent_1, Agent_2\dots Agent_N \rbrace$,  \textit{$EtatsDuMonde$} = $\lbrace Etat_1, Etat__2 \dots Etat__M \rbrace$}
\textit{$AGENTS$} = $\lbrace Agent_{1}, Agent_{2}\dots Agent_{M} \rbrace$,\\ \quad \quad \enspace \qquad \textit{$EtatsDuMonde$} = $\lbrace Etat_{1}, Etat_{2}\dots Etat_{N} \rbrace$}
\KwOut{
 $\lbrace Agent_1, Agent_2\dots Agent_K \rbrace$}
\BlankLine 
\Begin

~~\\
$Ensemble \enspace SousEnsembles \gets SousEnsebles(EtatsDuMonde)$ ~~\\
\tcc{La fonction SousEnsebles permet de générer tous les sous ensembles de l'ensemble donné en paramètre}

$Agents \enspace AgentsPréparés$
\\\Pour{$i \gets 0$ \KwTo $N$}{
$massSom \gets 0;$
\\\Si{$Agent(i).désactivé $}{
$ignorer$ \;
}
\Pour{$Chaque \enspace hypothèse \enspace de \enspace Agent$}{
\Si{$hypothèse \enspace \ne \enspace \Omega$}{
$Agent(i).Ajouter(hypothèse.id,hypothèse.masse \times Agent(i).Fiabilité) ;$
$ massSom \gets massSom + hypothèse.masse ;$ 
}
}

\Pour {$Chaque \enspace ensemble \enspace de \enspace SousEnsembles$}{
\Si{$ensemble \ne \varnothing \enspace \&\& \enspace Agent.hypothèse.Existe(ensemble)$}{
\Si{$ensemble = \Omega$}{
$Agent.Ajouter(ensemble.id,(1-massSom )\times Agent(i).Fiabilité) ;$
}
$Agent.Ajouter(ensemble.id,0);$
}

\Si{$ensemble$ $=$ $\Omega$}{
$Agent.Ajouter(\Omega.id,1-Fiabilité \times (\Omega.masse+ massSom);$ 
}
}
}
$AgentsPréparés.Ajouter(Agent);$
\\\Retourner{$AgentsPréparés$}

\End
\end{algorithm}


\section{Appel de procédures de fusion et de croyance}

Grâce à cet algorithme, nous pouvons appeler les méthodes de combinaison en passant les agents deux par deux en paramètres, de ce fait on peut fusionner un nombre plus de deux connaissances d'agents.\\

\begin{algorithm}[H]
\caption{Appel de procédures de fusion et de croyance}
\BlankLine
\KwIn{
$AGENTS = \lbrace Agent_{1}, Agent_{2}\dots Agent_{N} \rbrace $}
\KwOut{$Agent$}
\BlankLine 
\Begin
\\
\Si{$N < 1 $}{
$Agent AgentTemporaire \gets AGENTS(1);$
\\
\Pour{$i \gets 2$ \KwTo $N$}{
\Switch{$Méthode$}{
\Case{$Dempster-Shafer$}{
$AgentTemporaire \gets MultiAgentDempsterShafer(AgentTemporaire,AGENTS(i));$
}
\Case{$Dubois-Prade$}{
$AgentTemporaire \gets MultiAgentDuboisPrade(AgentTemporaire,AGENTS(i));$
}
\Case{$Smets$}{
$AgentTemporaire \gets MultiAgentSmets(AgentTemporaire,AGENTS(i));$
}
\Case{$Yager$}{
$AgentTemporaire \gets MultiAgentYager(AgentTemporaire,AGENTS(i));$
}
}
}
\vspace{1em}
\Si{$N =< 1 $}{
$CalculCroyancePlausibilité(Agent);$
}
}
\Retourner{$AgentsPréparés$}
\end{algorithm}
\vspace{2em}
\section{Procédure de combinaison de d'information}
\vspace{1em}
\begin{algorithm}[H]
\caption{Méthode de combinaison Dempster-Shafer}
\BlankLine
\KwIn{
$AGENT1 = \lbrace \lbrace hypothèse_{1},masse_{1} \rbrace \lbrace hypothèse_{2},masse_{2} \rbrace \dots \lbrace hypothèse_{n},masse_{n} \rbrace \rbrace $,$\newline AGENT2 = \lbrace \lbrace hypothèse_{1},masse_{1} \rbrace \lbrace hypothèse_{2},masse_{2} \rbrace \dots \lbrace hypothèse_{m},masse_{m} \rbrace \rbrace $}
\KwOut{$AGENTRes =\newline \lbrace \lbrace hypothèse_{1},masse_{1} \rbrace \lbrace hypothèse_{2},masse_{2}  \rbrace \dots \lbrace hypothèse_{l},masse_{l} \rbrace \rbrace $}
\BlankLine 
\Begin
$AGENTRes.Ajouter(AGENT1,0)$
$AGENTRes.Ajouter(AGENT2,0)$
 \tcc{Ajouter touts les éléments de $AGENT1$ et $AGENT2$  avec une masse $= 0$ }
$ k \gets 0;$
$\newline$
\Pour{$i \gets 1$ \KwTo $N$}{
\Pour{$j \gets 1$ \KwTo $M$}{
\Si{$hypothèse_{i} \cap hypothèse_{j} = \varnothing $}{
$K \gets K + AGENT1.masse(i) \times AGENT2.masse(j);$
}
$AGENTRes.(hypothèse_{i} \cap hypothèse_{j}).masse \gets AGENTRes.(hypothèse_{i} \cap hypothèse_{j}).masse + hypothèse_{i}.masse \times hypothèse_{j}.masse ;$
}

\vspace{1em}
}
$k \gets 1-k;$ \\
\Pour{$k \gets 1$ \KwTo $L$}{
$AGENTRes(k).masse \gets AGENTRes(k).masse \times (1/k);$
}
\Retourner{$AGENTRes$}
\end{algorithm}
$\newline$
$\\ $Dans les algorithmes de combinaison qui suit une grande partie redondante n'est pas presenté car elle est déjà dans l'algorithme combinaison Dempster-Shafer, nous ne présenterons que les paries différentes.
$\newline$
$\newline$
$\newline$
\begin{algorithm}[H]
\caption{Méthode de combinaison Dubois-Prade}
\setcounter{AlgoLine}{3}
$AGENTRes.(hypothèse_{i} \cap hypothèse_{j}).masse \gets AGENTRes.(hypothèse_{i} \cap hypothèse_{j}).masse + hypothèse_{i}.masse \times hypothèse_{j}.masse ;$

$AGENTRes.(hypothèse_{i} \cup hypothèse_{j}).masse \gets AGENTRes.(hypothèse_{i} \cup hypothèse_{j}).masse + hypothèse_{i}.masse \times hypothèse_{j}.masse ;$
\end{algorithm}
\vspace{3em}
\begin{algorithm}[H]
\setcounter{AlgoLine}{3}
\caption{Méthode de combinaison Smets}
\Si{$hypothèse_{i} \cap hypothèse_{j} = \varnothing $}{
$k \gets k + AGENTRes.(hypothèse_{i} \cap hypothèse_{j}).masse $}

$AGENTRes.(hypothèse_{i} \cap hypothèse_{j}).masse \gets AGENTRes.(hypothèse_{i} \cap hypothèse_{j}).masse + hypothèse_{i}.masse \times hypothèse_{j}.masse ;$
$AGENTRes.(\varnothing ).masse \gets k;$
\end{algorithm}
\vspace{3em}
\begin{algorithm}[H]
\setcounter{AlgoLine}{9}
\caption{Méthode de combinaison Yager}
\Si{$LeDernierApelle()$}{
$AGENTRes.(\varnothing ).masse \gets AGENTRes.(\varnothing ).masse + k;$}
 \tcc{La fonction LeDernierApelle permet de vérifier si c'est le dernier appel de la Méthode de combinaison en comptant le nombre d'agents }
$AGENTRes.(hypothèse_{i} \cap hypothèse_{j}).masse \gets AGENTRes.(hypothèse_{i} \cap hypothèse_{j}).masse + hypothèse_{i}.masse \times hypothèse_{j}.masse ;$
$AGENTRes.(\varnothing ).masse \gets k;$
\end{algorithm}
\vspace{3em}
Les Algorithmes ci dessus représente des implémentations sur les règles de combinaison présenté dans le premier chapitre, ces algorithmes nécessite beaucoup de ressource,ceci est causé par la méthode de générations des sous ensembles des états du monde dans l'étape de préparation des masses,cette méthode génère un nombre important d'ensemble :

Soit $E$ un ensemble à $n$ éléments. Alors, l'ensemble $\mathcal{P}(E)$ des parties de $E$ est fini, et a \textbf{$2^n$} éléments.

\vspace{4em}
Ce qui consomme un grand temps d'exécution en fonction de la grandeur d'états du monde, des instructions sont conçus afin  d'optimiser ces algorithmes ne sont pas présenté afin de rendre la présentation des algorithmes moins complexe et lisible.

\section{Procédure de calcul de croyance}
\vspace{1em}
\begin{algorithm}[H]
\caption{Calcul de Croyance et de Plausibilité}
\BlankLine
\KwIn{
$AGENT = \lbrace \lbrace hypothèse_{1},masse_{1} \rbrace \lbrace hypothèse_{2},masse_{2} \rbrace \dots \lbrace hypothèse_{n},masse_{n} \rbrace \rbrace $}
\KwOut{$AGENT = \lbrace \lbrace hypothèse_{1},masse_{1},CR_{1},PL_{1} \rbrace \lbrace hypothèse_{2},masse_{2},CR_{2},PL_{2}  \rbrace \dots$ \\$ \lbrace hypothèse_{n},masse_{n},CR_{n},PL_{n}  \rbrace \rbrace $}
\BlankLine 
\Begin

\Pour{$i \gets 1$ \KwTo $N$}{
$BL \gets 0;$
$PL \gets 0;$
$\newline$
\Pour{$j \gets 1$ \KwTo $N$}{
\Si{$hypothèse_{j} \subset hypothèse_{i} $}{
$BL \gets BL + masse_{i};$
}
}
\Pour{$j \gets 1$ \KwTo $N$}{
\Si{$hypothèse_{i} \cap hypothèse_{j} = \varnothing $}{
$PL \gets PL + masse_{j};$
}
}
\vspace{1em}

$AGENT(i).Ajouter(BL,PL);$

}
\Retourner{$AGENT$}
\end{algorithm}
\section{Procédure de calcul de décision}
\begin{algorithm}[H]
\caption{Méthode de calcul de décision}
\BlankLine
\KwIn{
$AGENT = \lbrace \lbrace hypothèse_{1},masse_{1},CR_{1},PL_{1} \rbrace \lbrace hypothèse_{2},masse_{2},CR_{2},PL_{2}  \rbrace \dots$ \\$ \lbrace hypothèse_{n},masse_{n},CR_{n},PL_{n}  \rbrace \rbrace $}
\KwOut{$ResDecision$}
\BlankLine 
$hypothèses = vecHypothèses$
\Pour{$i \gets 1$ \KwTo $N$}{
$DecPignistique \gets  0;$
\Pour{$i \gets 1$ \KwTo $N$}{
\Si{$Décision = Pignistique$}{
$DecPignistique = DecPignistique+ masse_i/hypothèse_{i}.nombreElements();$
}
}
\Si{$(hypothèse_i.nombreElements() = 1) \&\& (Décision = Pignistique)$}{
$vecHypothèses.Ajouter(hypothèse_{i},DecPignistique)$
}

}
\Switch{$Décision$}{
\Case{$Optimiste$}{
$ResDecision \gets MaxSingletonBL(AGENT);$
}
\Case{$Pessimiste$}{
$ResDecision \gets MaxSingletonPL(AGENT);;$
}
\Case{$Pignistique$}{
$ResDecision \gets vecHypothèses.Max();$
}
}
\end{algorithm}
\phantomsection
\addcontentsline{toc}{section}{Conclusion}
\section*{Conclusion}
