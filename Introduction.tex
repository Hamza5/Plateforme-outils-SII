\phantomsection
\addcontentsline{toc}{chapter}{Introduction générale}
\chapter*{Introduction générale}

Notre projet traite un sujet dans le domaine de l'intelligence artificielle. Il est dans le but de la
manipulation des connaissances incertaines.

Il est composé à partir de deux parties principales. D'abord, il sagit de réaliser un logiciel complet qui
implémente une des théories de ce domaine connue sous le nom de \emph{la théorie de Dempster-Shafer}. Ensuite,
il est dans l'objectif de créer une plateforme permettant de regrouper plusieurs outils dont le rôle de chaque outil est de
résoudre un problème d'incertain. Cette plateforme facilite l'utilisation de ces outils en offrant une interface
commune pour les exploiter.

Ce projet répond aux besoins des étudiants du Master 2 de la spécialité SII \footnote{Systèmes Informatiques Intelligents}
de notre département, dans leurs séances de TP. Il a comme but de compléter le manque d'un logiciel pour
effectuer les grands calculs requis par l'application de la théorie de Dempster-Shafer pour résoudre un
problème donné. En plus, il offre une interface pour tout les outils que les étudiants ont l'habitude de les utiliser.

Dans le premier chapitre nous expliquerons les concepts de la théorie de Dempster-Shafer et les règles de combinaison
et de décision associées. Dans le deuxième, nous parlons sur les théories de l'incertain qui sont implémentées
par les outils exploités à travers la plateforme. Dans le chapitre qui suit nous présentons quelques algorithmes que
nous avons conçu et implémenté dans notre logiciel. Enfin, le dernier chapitre est consacré pour la présentation de ce que
nous avons réalisé.