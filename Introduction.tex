\phantomsection
\addcontentsline{toc}{chapter}{Introduction générale}
\chapter*{Introduction générale}

Notre projet traite un sujet situé dans le domaine de l'intelligence artificielle. Il a comme but la
manipulation de connaissances incertaines.

Il présentes deux aspects principales. Il s'agit d'abord de réaliser un logiciel complet qui implémente l'une des
théories de ce domaine connue sous le nom de \emph{la théorie de Dempster-Shafer}. Ensuite, il y a l'objectif de
créer une plateforme permettant de regrouper plusieurs outils dont le rôle de chaqune est de résoudre un problème
d'incertain. Cette plateforme est destinée à faciliter l'utilisation de ces outils en offrant une interface
commune pour les exploiter.

C'est un projet qui répond à des besoins rencontrés au cours de séances de TP par les étudiants du Master 2 de la
spécialité SII \footnote{Systèmes Informatiques Intelligents} de notre département \footnote{Département informatique
de la faculté d'électronique et d'informatique de l'USTHB}. Il est nécessaire en effet de combler le manque d'un logiciel
pour pouvoir effectuer les grands calculs requis par l'application de la théorie de Dempster-Shafer et pour résoudre un
problème donné. De plus, il est conçu pour offrir une interface pour tout les outils que les étudiants ont l'habitude
d'utiliser.

Dans le premier chapitre nous expliquons les concepts de la théorie de Dempster-Shafer et les règles de combinaison
et de décision associées. Dans le deuxième, nous exposons les théories de l'incertain qui sont implémentées
par les outils exploités à travers la plateforme. Dans le chapitre qui suit nous présentons quelques algorithmes que
nous avons conçus et implémentés dans notre logiciel. Le dernier chapitre est consacré à la présentation de ce que
nous avons réalisé.