\phantomsection
\addcontentsline{toc}{chapter}{Introduction générale}
\chapter*{Introduction générale}
Les problématiques liées au domaine de l'intelligence artificielle tournent souvent autour de la représentation
des connaissances à modéliser et du raisonnement. C'est dans ce domaine que se situe notre sujet qui a comme but
de faciliter la manipulation de connaissances incertaines.

Il présente deux aspects principales. Il s'agit d'abord de réaliser un logiciel complet qui implémente l'une des
théories de ce domaine connue sous le nom de \emph{la théorie de Dempster-Shafer}. Ensuite, il a pour objectif de
créer une plateforme permettant de regrouper plusieurs outils développés en utilisant diverses théories de l'incertain
telles que la théorie de probabilités et la théorie des possibilités qui traitent les connaissances incertaines ainsi
que la théorie des sous ensembles flous pour traiter les connaissances imprécises. Cette plateforme est destinée à
faciliter l'utilisation de ces outils en offrant une interface commune pour les exploiter.

C'est un projet qui répond à des besoins rencontrés en cours et en séances de TP par les étudiants du Master 2 de la
spécialité SII \footnote{Systèmes Informatiques Intelligents} de notre département \footnote{Département informatique
de la faculté d'électronique et d'informatique de l'USTHB}. Il est nécessaire en effet de combler le manque de logiciels
pour pouvoir effectuer les grands calculs requis par l'application de la théorie de Dempster-Shafer et pour résoudre un
problème donné. De plus, il est conçu pour offrir une interface pour tout les outils que les étudiants ont l'habitude
d'utiliser.

Le mémoire est organisé comme suit.

Dans le premier chapitre nous expliquons les concepts de la théorie de Dempster-Shafer et les règles de combinaison
et de décision associées. Dans le deuxième chapitre, nous exposons les théories de l'incertain qui sont implémentées
par les outils exploités à travers la plateforme. Dans le troisième chapitre nous présentons quelques algorithmes que
nous avons conçu et implémenté dans notre logiciel. Le dernier chapitre est consacré à la présentation de ce que
nous avons réalisé.